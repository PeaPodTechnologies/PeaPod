By nature of its design, PeaPod will at least last three years at near 100\% functionality on minimal maintenance. This is achieved by self-monitoring component health, using easily-serviceable components, and providing smart notifications to the user when maintenance is needed.
For one, PeaPod is designed to be assembled by a single user with readily available hand tools. This means it can be disassembled, cleaned, and put back together by one person in a non-restrictive amount of time.

For another, the sensors used to monitor plant health and environment conditions allow PeaPod to perform self-diagnostics and notify the user when a part needs to be fixed or replaced. For example, if humidity readings fall despite power being applied to the humidification system, PeaPod will notify the user to check the humidification unit. If temperature readings fall despite power being applied to the thermoregulation system, PeaPod will notify the user to check the heating unit.

This said, every component in PeaPod has an expected lifespan over three years. From the LEDs (rated for 5 years) to the nozzles and fittings (high-quality brass), replacement monitoring is only needed as a backup. A replacement for each "active" part used in the entire assembly (i.e. non-housing, all moving/electrical/water parts) should be kept on board.

There are few moving parts, and no wear or lubrication required. The diaphragm pump is the most reliable and long-lasting pump variety \cite{diaphragm}. Fans are self-lubricating nylon. Solenoids and servos are rated for upwards of 5 years.