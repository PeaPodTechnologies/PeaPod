% \textbf{Customer-facing Food Service} %OpenComment does this make sense? feels a little weak/unfounded - NV

% A restaurant serving fresh produce needs either a local supplier or a substantial amount of outdoor space. Both of these are cost-prohibitive, and the latter is impossible in many urban situations. Local suppliers' high costs are the result of a few things:
% \begin{itemize}
%     \item Limited seasonal availability
%     \item Frequent transport need
%     \item High costs with little demand
% \end{itemize}
% PeaPod has the potential to reduce these barriers in a cyclic way. Partnerships between local suppliers and restaurants will provide these restaurants with space- and time-efficient PeaPod units with the purpose of generating both produce and data. The increase in produce will reduce the frequency at which suppliers need to make deliveries (reducing cost and spreading demand), while the data generated will let suppliers maximize output, quality, and longevity. Over time, this increases efficiency to the point where local suppliers can provide produce at a lower price, increasing the presence of fresh, local, produce in cities worldwide.

% Another food service application of PeaPod is the globalization of otherwise endemic crops. For example, Wasabi is a difficult-to-grow Japanese crop that is near-impossible to grow overseas (save a few recent commercial developments). PeaPod units in restaurants solve this issue by re-creating the necessary precise conditions and eliminating barriers in skill and time that otherwise impact Wasabi's production and consumption overseas. And, of course, PeaPod's data collection capabilities can be harnessed by farmers to discover more safe, efficient, and adaptable methods of producing crops such as these.

\textbf{Agriculture-as-a-Service}

PeaPod’s modularity and ease of storage can turn unused city spaces to PeaPod farms. With fresh produce in local areas, PeaPod can be a direct food system to paying customers. A subscription service would provide patrons with fresh produce without a middle man. By eliminating transport, distribution, and grocery stores, PeaPod creates fresher, better produce for the general public at a lower cost.

\textbf{Crowd-Sourced Research}

PeaPod's automation is unique in the research space, allowing for autonomous, off-site research. Universities save costs related to space and energy usage by subsidizing PeaPods to individuals, schools, or even restaurants. Users receive sets of parameters to grow crops with, sending data back to the institution and using the produce at the cost of space and energy. The result is a massive dataset from identical conditions in different places, verified by comparison with devices conducting the same tests.

This is an effective tool in climate change famine aid. By predicting conditions in at-risk areas, researchers can conduct tests ahead of time to determine what seeds, traits, and care parameters are most effective for certain conditions. This also informs development of seeds specialized for extreme climates, letting areas counteract food scarcity by having a variety of options prepared ahead of time.

\textbf{De-centralized Production}

Many crops are endemic to certain climates, making global transport necessary to for foreign markets. This reduces freshness, necessitates preservatives, and increases the carbon footprint of agriculture. By upscaling PeaPod technology to a farm scale, climate-bound crops can be produced anywhere. This creates regional farms of global variety, making it easier to have a local food diet.

PeaPod's form factor makes it a viable tool for at-home production, either in cities or off-grid. With only a solar power source, water, and a compact supply of nutrient solutions, users can sustain crops even through winter without travelling for nutrients and supplies.

% \textbf{Blockchain and "PlantCoin" Cryptocurrency Mining}

% In the same way current cryptocurrencies reward nodes for validating all information on a blockchain, a PeaPod network would reward nodes (individual PeaPods) for validating data produced by other nodes. When tens or hundreds of units produce statistically similar results in independent trials using the same parameters, all units are rewarded for their contribution to the network. This process is commonly known as "mining". This allows consumers the option to operate a PeaPod as a passive source of income, with the unit paying for itself by both producing food and mining a "PlantCoin" cryptocurrency.

% This also incentivizes the network to find critical points in the n-dimensional set of data that would otherwise not be checked. For example, few users would intentionally grow plants in drought conditions at this will produce subpar produce. But, by incentivizing users to fill sparse sections of data, we can generate information critical to developing crops for territories affected by climate change.

%we love this idea but idt nasa cares

\textbf{Food Infrastructure Micro-Loans}

For many, finding fresh produce is a struggle whilst growing your own is prohibitively expensive. Micro-loan platforms have attempted to solve this by letting donors fund an interest-free loan for technology/infrastructure which then pays the loan as a percentage of its surplus. 

Unfortunately, these are only feasible for individuals in rural areas with arable land and climate.

PeaPod brings this solution to low-income urban areas with a platform for donors to micro-loan PeaPod units and inputs. The user feeds themselves and sells surplus, while a percentage of sales go to the interest-free loan. Once paid, PeaPod continues to produce food while sales fund its operation. 

This creates permanent, self-sustaining agricultural infrastructure that pays for itself as it grows, requiring little initial capital. This means entire farms throughout high density buildings generating yield with little lost space and almost no labour.