PeaPod's input stability is maximized by a variety of design choices, the sum of which give them a shelf life above the three-year mark of a mission. Since the system doses nutrients automatically and at a high-degree of precision, nutrient solution can be stored at a much greater density than would be possible with manual mixing. This minimizes degradation and loss of quality while reducing the space needed to store the solutions. Solutions are also stored in insulated, opaque container that minimize fluctuations in temperature and light that could stimulate compound degradation.

Outputs will have a shelf life that is, in worst case, comparable to fresh produce grown outdoors. More realistically, crops are expected to last longer as a result of a lack of pests, disease, and optimization of plant characteristics for ambient conditions, as well as eliminating time and transport between cultivation and consumption. Finally, PeaPod can let users grow crops on a rotation, providing a steady supply of fresh produce that will not need to be stored for particularly long periods of time, thus circumventing some of the restrictions posed by growing fresh crops.