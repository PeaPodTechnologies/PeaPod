There are several considerations when examining the acceptability of the products of our system:

\begin{enumerate}
    \item Produce is not only eaten fresh, but also forms the basis for an innumerable variety of combined and prepared foods (i.e. fresh tomatoes vs. tomato sauce);
    \item When considering prepared derivatives of the food products, the quality of the preparation is a key factor in acceptability. As such, proper care in training and is to be taken;
    \item The products formed by the system (and their properly prepared derivatives) are not exceptional or novel. They are the same plant-based foods grown, consumed, and \textbf{accepted} terrestrially, just grown in a more efficient and controlled way. As such, their acceptability is determined to be of \textbf{equal or greater value};
    \item Plant-environment optimization can be targeted not only at nutritional value or efficiency, but also at acceptability. The feedback can be gathered either through crew Hedonic rating (i.e. tomatoes grown in environment ABC rate X in appearance, Y in aroma, etc.) or more sophisticated analysis (i.e. computer vision analysis of color/size/shape for appearance, tissue concentrations of various aroma/flavor compounds); %TODO: cite some times when this has been performed, i.e. MIT basil study
\end{enumerate}