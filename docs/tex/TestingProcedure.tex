\section{Testing Procedure}
% Per-requirement testing plan
% Processes, measurements, experiments
% NOTE: Often, acceptable ranges in results depend on specifics of the technology

\subsection{Acceptability}
% Are people willing to eat output? Incl. appearance, aroma, flavor, and texture
% Specific preparation and/or incorporation of output (formulation) for tasting

% Suggestion: Double-blind volunteer study?

\subsection{Safety of Process}
% SPECIFICALLY food production area
% Chemical hazards - Toxins, heavy metals (As, Cd, Hg, Pb, etc.)
% Biohazards - total aerobic plate count, ATP testing of food contact surfaces
% TODO: What do those^ mean? How do we test them?

\subsection{Safety of Outputs}
% SPECIFICALLY food outputs
% Biohazards - total aerobic count, specific pathogens (enterobacteriaceae, salmonella, yeasts, molds, E. coli, Listeria, etc.)

\subsection{Resource Outputs}
% Nutritional analysis - macro- and micro-nutrients

\subsection{Reliability and Stability of Outputs}
% Biohazards - total aerobic plate count, enterobacteriaceae, yeasts, molds

\section{Sample Collection Procedure and Schedule}
% Days/cycles of operation before sample collection?
% Incl. sample collection (size, timing, quantity), packaging, shipping