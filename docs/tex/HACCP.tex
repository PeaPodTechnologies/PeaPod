\section{Hazard Analysis and Critical Control Point (HACCP) Plan}
% Specific focus on the safety of the system's processes and food outputs

% HACCP is a systematic approach to the identification, evaluation, and control of food safety hazards. A HACCP Plan is the written document which is based upon the principles of HACCP and which delineates the procedures to be followed.
% The Teams’ HACCP Plan should follow the principles and guidelines of a HACCP Plan as described by the U.S. National Advisory Committee on Microbiological Criteria for Foods (NACMCF) and the associated prerequisite programs (where applicable).
% The Basic HACCP Plan desired at this Progress Report stage (May 31 deadline) is only expected to be a starting point for the final Basic HACCP Plan expected to be a part of the final report (due in January 2023).

% Def'n of terms: https://www.fda.gov/food/hazard-analysis-critical-control-point-haccp/haccp-principles-application-guidelines#princ
% HACCP reference: https://spinoff.nasa.gov/moon-landing-food-safety?utm_source=TWITTER&utm_medium=KathyLueders&utm_campaign=NASASocial&linkId=141839394

\subsection{Food Production System Description}
% Incl. flowchart


\subsection{Critical Points}
%assembly: only approved parts that have been checked for leaks, contamination, materials, air-tightness

%seeds: vetted and tested breed from an approved supplier, not opened until moment of planting, food safety steps followed while doing so (hand washing, gloves, masks, hairnet...)

%growth medium: tested for bacteria/pests, handled carefully before installation

%system inputs: filtered air, clean water, properly sourced nutrients, all tested by appropriate standards

%maintenance: plants properly isolated when non-food safe materials are present---i.e. if LED boards need to be swapped, if something needs to be greased, if something needs to be glued

%harvesting: all food safety guidelines to be followed - hand washing, gloves, masks, washing, etc.

%re-planting: checking for no old growth medium, sanitizing growth tray with food-safe sanitizer

\subsubsection{Critical Point A}
% Repeat for all control points

\textbf{Hazard Description}
% aka hazard analysis, discover CCPs

\textbf{Critical Limits}
% What is the safe range AND the failure conditions for the CCP

\textbf{Monitoring Procedures}
% How do we know the state of the CCP

\textbf{Deviation Procedures}
% aka corrective actions
% How do we keep the CCP within range?

\textbf{Associated Documents}
% record-keeping and documentation procedures

% TODO: Verification Procedures?

\subsection{Standard Test Record}
% TODO: What exactly is this?

\subsubsection{Purpose and Summary}

\subsubsection{Safety and Quality}

\subsubsection{Test Processes}

\textbf{Preparation of Inputs}\\


\textbf{Verification}\\


\textbf{Setup, Maintenance, and Collection Protocols}\\


\textbf{Storage}\\


\textbf{Cleanup and Turnover}\\


\subsubsection{Closeout}

