\documentclass{../tex/report}
\usepackage{setspace} % Setting line spacing
\usepackage{ulem} % Underline
\usepackage{caption} % Captioning figures
\usepackage{subcaption} % Subfigures
\usepackage{geometry} % Page layout
\usepackage{multicol} % Columned pages
\usepackage{array,etoolbox}
\usepackage{fancyhdr}
\usepackage{enumitem}
\usepackage[toc,page]{appendix}
\setlist{noitemsep}

% Page layout (margins, size, line spacing)
\geometry{letterpaper, left=1in, right=1in, bottom=1in, top=1in}
\setstretch{1}

% Headers
\pagestyle{fancy}
\lhead{PeaPod - Final Report}
\rhead{PeaPod Technologies Inc.}

\begin{document}

\begin{titlepage}
    \begin{center}
        \vspace*{1.2cm}

        \textbf{\large{PeaPod - Final Report}}

        \vspace{0.5cm}

        NASA/CSA Deep Space Food Challenge Phase 2

        \vfill
        \small{
    \textbf{Jayden Lefebvre - Founder, Lead Engineer}\\
    Port Hope, ON, Canada\\
    \vspace{.5cm}
    \textbf{Nathan Chareunsouk - Design Lead}\\Toronto, ON, Canada\\
    \vspace{.5cm}
    \textbf{Navin Vanderwert - Design Engineer}\\
    BASc Engineering Science (Anticipated 2024), University of Toronto\\
    Toronto, ON, Canada\\
    \vspace{.5cm}
    \textbf{Jonas Marshall - Electronics Engineer}\\
    BASc Computer Engineering (Anticipated 2024), Queen's University\\
    Kingston, ON, Canada
}

\vspace{1cm}

Primary Contact Email: contact@peapodtech.com
        \vspace{.75cm}

        Revision 0.1\\
        PeaPod Technologies Inc.\\
        January 10th, 2022

    \end{center}
\end{titlepage}

\thispagestyle{plain}

\tableofcontents
\newpage

\section{Summary}
% How does it work?
% What is novel, sustainable, and innovative?
% What foods does it create?
% How are you minimizing inputs and maximizing outputs?

\subsection{Process Description}
% Please give a brief update of your food production system’s process (including maintenance and cleaning):
% Fully-detailed process description (incl. setup, operation, maintenance, and cleaning), flow diagram for each
% maximum 5000 characters
\subsubsection{Setup}
\subsubsection{Operation}
\subsubsection{Maintenance}
\subsubsection{Cleaning}

\section{Inputs}

\section{Outputs}

\documentclass{report}
\usepackage{setspace} % Setting line spacing
\usepackage{ulem} % Underline
\usepackage{caption} % Captioning figures
\usepackage{subcaption} % Subfigures
\usepackage{geometry} % Page layout
\usepackage{multicol} % Columned pages
\usepackage{array,etoolbox}
\usepackage{fancyhdr}
\usepackage{enumitem}
\usepackage[toc,page]{appendix}
\setlist{noitemsep}

% Page layout (margins, size, line spacing)
\geometry{letterpaper, left=1in, right=1in, bottom=1in, top=1in}
\setstretch{1}

% Headers
\pagestyle{fancy}
\lhead{PeaPod - Testing Plan}
\rhead{PeaPod Technologies Inc.}

\begin{document}

\begin{titlepage}
    \begin{center}
        \vspace*{1.2cm}

        \textbf{\large{PeaPod - Testing Plan}}

        \vspace{0.5cm}

        NASA/CSA Deep Space Food Challenge Phase 2

        \vfill
        \small{
    \textbf{Jayden Lefebvre - Founder, Lead Engineer}\\
    Port Hope, ON, Canada\\
    \vspace{.5cm}
    \textbf{Nathan Chareunsouk - Design Lead}\\Toronto, ON, Canada\\
    \vspace{.5cm}
    \textbf{Navin Vanderwert - Design Engineer}\\
    BASc Engineering Science (Anticipated 2024), University of Toronto\\
    Toronto, ON, Canada\\
    \vspace{.5cm}
    \textbf{Jonas Marshall - Electronics Engineer}\\
    BASc Computer Engineering (Anticipated 2024), Queen's University\\
    Kingston, ON, Canada
}

\vspace{1cm}

Primary Contact Email: contact@peapodtech.com
        \vspace{.75cm}

        Revision 0.1\\
        PeaPod Technologies Inc.\\
        January 9th, 2022

    \end{center}
\end{titlepage}

\thispagestyle{plain}

\tableofcontents
\newpage

\section{Testing Procedure}
% Per-requirement testing plan
% Processes, measurements, experiments
% NOTE: Often, acceptable ranges in results depend on specifics of the technology

\subsection{Acceptability}
% Are people willing to eat output? Incl. appearance, aroma, flavor, and texture
% Specific preparation and/or incorporation of output (formulation) for tasting

% Suggestion: Double-blind volunteer study?

\subsection{Safety of Process}
% SPECIFICALLY food production area
% Chemical hazards - Toxins, heavy metals (As, Cd, Hg, Pb, etc.)
% Biohazards - total aerobic plate count, ATP testing of food contact surfaces
% TODO: What do those^ mean? How do we test them?

\subsection{Safety of Outputs}
% SPECIFICALLY food outputs
% Biohazards - total aerobic count, specific pathogens (enterobacteriaceae, salmonella, yeasts, molds, E. coli, Listeria, etc.)

\subsection{Resource Outputs}
% Nutritional analysis - macro- and micro-nutrients

\subsection{Reliability and Stability of Outputs}
% Biohazards - total aerobic plate count, enterobacteriaceae, yeasts, molds

\section{Sample Collection Procedure and Schedule}
% Days/cycles of operation before sample collection?
% Incl. sample collection (size, timing, quantity), packaging, shipping

\section{Hazard Analysis and Critical Control Point (HACCP) Plan}
% Def'n of terms: https://www.fda.gov/food/hazard-analysis-critical-control-point-haccp/haccp-principles-application-guidelines#princ
% HACCP reference: https://spinoff.nasa.gov/moon-landing-food-safety?utm_source=TWITTER&utm_medium=KathyLueders&utm_campaign=NASASocial&linkId=141839394

\subsection{Food Production System Description}
% Incl. flowchart


\subsection{Critical Points}
%assembly: only approved parts that have been checked for leaks, contamination, materials, air-tightness

%seeds: vetted and tested breed from an approved supplier, not opened until moment of planting, food safety steps followed while doing so (hand washing, gloves, masks, hairnet...)

%growth medium: tested for bacteria/pests, handled carefully before installation

%system inputs: filtered air, clean water, properly sourced nutrients, all tested by appropriate standards

%maintenance: plants properly isolated when non-food safe materials are present---i.e. if LED boards need to be swapped, if something needs to be greased, if something needs to be glued

%harvesting: all food safety guidelines to be followed - hand washing, gloves, masks, washing, etc.

%re-planting: checking for no old growth medium, sanitizing growth tray with food-safe sanitizer

\subsubsection{Critical Point A}
\textbf{Hazard Description}
% aka hazard analysis, discover CCPs

\textbf{Critical Limits}
% What is the safe range AND the failure conditions for the CCP

\textbf{Monitoring Procedures}
% How do we know the state of the CCP

\textbf{Deviation Procedures}
% aka corrective actions
% How do we keep the CCP within range?

\textbf{Associated Documents}
% record-keeping and documentation procedures

% TODO: Verification Procedures?

\subsubsection{Critical Point ...}

\subsection{Standard Test Record}
% TODO: What exactly is this?

\subsubsection{Purpose and Summary}

\subsubsection{Safety and Quality}

\subsubsection{Test Processes}

% Preparation of Inputs
% Verification
% Setup, Maintenance, and Collection Protocols
% Storage
% Cleanup and Turnover

\subsubsection{Closeout}

\newpage

% References
\bibliographystyle{IEEEtran}
\bibliography{references}
\end{document}

\section{Results}

\subsection{Acceptability}
% Are people willing to eat output? Incl. appearance, aroma, flavor, and texture
% Specific preparation and/or incorporation of output (formulation) for tasting

% Suggestion: Double-blind volunteer study?

\subsection{Safety of Process}
% SPECIFICALLY food production area
% Chemical hazards - Toxins, heavy metals (As, Cd, Hg, Pb, etc.)
% Biohazards - total aerobic plate count, ATP testing of food contact surfaces
% TODO: What do those^ mean? How do we test them?

\subsection{Safety of Outputs}
% SPECIFICALLY food outputs
% Biohazards - total aerobic count, specific pathogens (enterobacteriaceae, salmonella, yeasts, molds, E. coli, Listeria, etc.)

\subsection{Resource Outputs}
% Nutritional analysis - macro- and micro-nutrients

\subsection{Reliability and Stability of Outputs}
% Biohazards - total aerobic plate count, enterobacteriaceae, yeasts, molds

\section{Hazard Analysis and Critical Control Point (HACCP) Plan}
% Def'n of terms: https://www.fda.gov/food/hazard-analysis-critical-control-point-haccp/haccp-principles-application-guidelines#princ
% HACCP reference: https://spinoff.nasa.gov/moon-landing-food-safety?utm_source=TWITTER&utm_medium=KathyLueders&utm_campaign=NASASocial&linkId=141839394

\subsection{Food Production System Description}
% Incl. flowchart


\subsection{Critical Points}
%assembly: only approved parts that have been checked for leaks, contamination, materials, air-tightness

%seeds: vetted and tested breed from an approved supplier, not opened until moment of planting, food safety steps followed while doing so (hand washing, gloves, masks, hairnet...)

%growth medium: tested for bacteria/pests, handled carefully before installation

%system inputs: filtered air, clean water, properly sourced nutrients, all tested by appropriate standards

%maintenance: plants properly isolated when non-food safe materials are present---i.e. if LED boards need to be swapped, if something needs to be greased, if something needs to be glued

%harvesting: all food safety guidelines to be followed - hand washing, gloves, masks, washing, etc.

%re-planting: checking for no old growth medium, sanitizing growth tray with food-safe sanitizer

\subsubsection{Critical Point A}
\textbf{Hazard Description}
% aka hazard analysis, discover CCPs

\textbf{Critical Limits}
% What is the safe range AND the failure conditions for the CCP

\textbf{Monitoring Procedures}
% How do we know the state of the CCP

\textbf{Deviation Procedures}
% aka corrective actions
% How do we keep the CCP within range?

\textbf{Associated Documents}
% record-keeping and documentation procedures

% TODO: Verification Procedures?

\subsubsection{Critical Point ...}

\subsection{Standard Test Record}
% TODO: What exactly is this?

\subsubsection{Purpose and Summary}

\subsubsection{Safety and Quality}

\subsubsection{Test Processes}

% Preparation of Inputs
% Verification
% Setup, Maintenance, and Collection Protocols
% Storage
% Cleanup and Turnover

\subsubsection{Closeout}

\subsection{Materials}
\subsubsection{System}

\textbf{Automation}\\

\begin{table}[!h]
    \centering
    \begin{tabular}{|c|l|l|l|c|}
    \hline
        Index   & Manufacturer Part Number  & Manufacturer Name         & Description                       & Quantity  \\ \hline
        1       & A000005                   & Arduino                   & ARDUINO NANO ATMEGA328 EVAL BRD   & 1         \\ \hline
        2       & S404GSEJ6-U3000-3         & "Delkin Devices, Inc."    & 4GB MLC MICROSD CARD (-25C - +85  & 1         \\ \hline
        3       & 61304021121               & Würth Elektronik          & CONN HEADER VERT 40POS 2.54MM     & 1         \\ \hline
        4       & SC0510                    & Raspberry Pi              & ZERO 2 W                          & 1         \\ \hline
        5       & DMN2005K-7                & Diodes Incorporated       & MOSFET N-CH 20V 300MA SOT23-3     & 2         \\ \hline
        6       & RC0603FR-0710KL           & YAGEO                     & RES 10K OHM 1\% 1/10W 0603        & 5         \\ \hline
        7       & 4484                      & Adafruit Industries LLC   & MINI PITFT 1.3 FOR RASPBERRY PI   & 1         \\ \hline
        8       & 5055670271                & Molex                     & CONN HEADER SMD R/A 2POS 1.25MM   & 2         \\ \hline
        9       & 5055670471                & Molex                     & CONN HEADER SMD R/A 4POS 1.25MM   & 5         \\ \hline
        10      & 5055670871                & Molex                     & CONN HEADER SMD R/A 8POS 1.25MM   & 3         \\ \hline
        11      & 5055670681                & Molex                     & CONN HEADER SMD R/A 6POS 1.25MM   & 3         \\ \hline
    \end{tabular}
    \caption{Automation system electronic components.}
    \label{tab:automation_components}
\end{table}

In addition, 1x \textit{Automation Motherboard PCB}: 2 Layers, 1 oz. Copper, 1.6mm Thickness, Suggested: HASL Finish (Lead-Free), White PCB, Black Silkscreen

\textbf{Housing}\\


\textbf{Aeroponics}\\


\textbf{Leaf-Zone Thermoregulation}\\


\textbf{Humidification}\\


\textbf{Dehumidification}\\


\textbf{Gas Composition Regulation and Exchange}\\


\textbf{Lighting}\\


\subsubsection{Inputs}
% Supply inputs (water, power, network), consumable inputs (pH/nutrient solutions, dehumidification cartridge)

\textbf{Supply Inputs}
\begin{itemize}
    \item \textit{Water}: reverse-osmosis, ambient
    \item \textit{Power}: 120V 60Hz AC\footnote{The power supply can be altered to suit a variety of power inputs (i.e. DC)}
    \item \textit{Network}: ethernet or wireless, optional
\end{itemize}

\textbf{Consumable Inputs}
\begin{itemize}
    \item \textit{Nutrient/pH Adjusment Solutions}: pouches
    \item \textit{Dehumidification Cartridge}: recharged
\end{itemize}

\subsubsection{Outputs}
% Food outputs(?), by-products/waste (waste water from flushing, dehumidification cartridge)

\textbf{Food Outputs}\\


\textbf{By-Products \& Waste}\\


\subsubsection{Maintenance}

\textbf{Spare Components}\\


\textbf{Tools}\\


\subsubsection{Cleaning}

\textbf{Soaps}\\


\textbf{Disinfectants}\\


\textbf{Tools}\\



\subsection{Demonstration and Testing Materials}

\section{Terrestrial Application}
% Preliminary concrete step-by-step plan on how the tech could improve terrestrial food production
% Identify gaps/deficient areas in existing processes, and explain how this system would fill
% Incl. summary of the problem and how we solve it, list of potential engagements with communities (local and otherwise), co-creation strategy (?) for a future implementation

\section{Scaling Potential}
% Main steps for building the full-scale technology (for phase 3)

\newpage

% References
\bibliographystyle{IEEEtran}
\bibliography{references}
\end{document}