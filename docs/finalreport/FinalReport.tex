\documentclass{../tex/report}
\usepackage{setspace} % Setting line spacing
\usepackage{ulem} % Underline
\usepackage{caption} % Captioning figures
\usepackage{subcaption} % Subfigures
\usepackage{geometry} % Page layout
\usepackage{multicol} % Columned pages
\usepackage{multirow}
\usepackage{array,etoolbox}
\usepackage{fancyhdr}
\usepackage{enumitem}
\usepackage{tabularx}
\usepackage[toc,page]{appendix}
\setlist{noitemsep}

% Page layout (margins, size, line spacing)
\geometry{letterpaper, left=1in, right=1in, bottom=1in, top=1in}
\setstretch{1}

% Headers
\pagestyle{fancy}
\lhead{PeaPod - Final Report}
\rhead{PeaPod Technologies Inc.}

\begin{document}

\begin{titlepage}
    \begin{center}
        \vspace*{1.2cm}

        \textbf{\large{PeaPod - Final Report}}

        \vspace{0.5cm}

        NASA/CSA Deep Space Food Challenge Phase 2

        \vfill
        \small{
    \textbf{Jayden Lefebvre - Founder, Lead Engineer}\\
    Port Hope, ON, Canada\\
    \vspace{.5cm}
    \textbf{Nathan Chareunsouk - Design Lead}\\Toronto, ON, Canada\\
    \vspace{.5cm}
    \textbf{Navin Vanderwert - Design Engineer}\\
    BASc Engineering Science (Anticipated 2024), University of Toronto\\
    Toronto, ON, Canada\\
    \vspace{.5cm}
    \textbf{Jonas Marshall - Electronics Engineer}\\
    BASc Computer Engineering (Anticipated 2024), Queen's University\\
    Kingston, ON, Canada
}

\vspace{1cm}

Primary Contact Email: contact@peapodtech.com
        \vspace{.75cm}

        Revision 0.2\\
        PeaPod Technologies Inc.\\
        January 31st, 2023

    \end{center}
\end{titlepage}

\thispagestyle{plain}

\tableofcontents
\newpage

\section{Technology Description and Progress}
% How does it work?
% What is novel, sustainable, and innovative?
% What foods does it create?
% How are you minimizing inputs and maximizing outputs?

% With a focus on progress or changes made during Phase 2 of the challenge, describe what the food production technology is, what it does, how it functions, and how the crew will interact with it. Include descriptions of major hardware components and processes. (limit 4000 characters).

\subsection{Process Description}
% Please give a brief update of your food production system’s process (including maintenance and cleaning):
% Fully-detailed process description (incl. setup, operation, maintenance, and cleaning), flow diagram for each
% maximum 5000 characters
\subsubsection{Setup}
\subsubsection{Operation}
\subsubsection{Maintenance}
\subsubsection{Cleaning}

\section{Adherence to Constraints}

\subsection{Volume}
% 100chars

\textbf{Constraint}: \textit{Fits through 1.07m x 1.90m doorway; W < 1.820m, D < 2.438m, H < 2.591m; V<= 2m${}^3$}

\textbf{Standard Expanded PeaPod}: 4x3x1 units (0.5m on all sides) + control module = 2m wide x 1.7m tall x .5m deep = 1.5m${}^3$ (< 2)

With width treated as depth for the purposes of the considerations of the "room size" constraint, the Standard meets this constraint.

\subsection{Power}
% 1000chars

\textbf{Constraint}: \textit{Avg < 1500 W, Peak < 3000 W}

% For calculations and justification, see preliminary calculations Appendix \ref{app:power}.

\textbf{Total Power Consumption}: 1,284W

\subsection{Water Consumption}
% limit 1000 characters
% CNet = (Initial water input + water added over time + water used for cleaning – condensed/ transpired water provided back to crew)

\textbf{Unconstrained}

% For calculations and justification, see preliminary calculations Appendix \ref{app:water}.

\textit{Humidification}: CNet = \textbf{500mL initial + 50mL per hour}

\textit{Aeroponics}: CNet = \textbf{1.25L primed + 1.2L per hour}.

\subsection{Mass}
% limit 1000chars

% For part breakdown, calculations, and justification, see preliminary calculations Appendix \ref{app:mass}.

Total: \textbf{70kg}

\subsection{Data Connection}
% limit 1000 characters

% \textit{Automation}: All of PeaPod's operation is automated, save for a select few setup, maintenance, and harvest tasks. This is controlled by a central computer, which uses a "program" to enact the desired environment at each point in time throughout the plant life cycle. 
\textit{Remote Control}: The program may be changed \uline{instantaneously} as an \textbf{appended instruction set} with immediate effect.
% \textit{Feedback}: The feedback sensors provide the computer with information that will influence the control it exerts. For example, if the program indicates the leaf zone temperature should be set to 22 degrees C, the computer would apply greater power to the heater if the current temperature was 18 degrees C as opposed to 21 degrees C. This forms a control loop for each parameter, relying on one of many control functions (bang-bang, PID, etc.).

\textit{Data Presentation}: Plant and environment data can be viewed with \uline{live updates and video feed}.

\subsection{Crew Time}

\textbf{Constraint}: \textit{4 hrs/week}

For calculations and justification, see preliminary calculations Appendix.

Total setup process time (2 trays per unit, 12 units, 1 CM): \textbf{17.5 hours} (\textit{one person}) or \textbf{4.5 hours} (\textit{crew of 4})

Total maintenance time per week: \textbf{1-2 hours} (depending on program)

\subsection{Operational Constraints} 

\textbf{Constraint}: \textit{Terrestrial: gravity (9.81 m/s${}^2$), ambient atmospheric pressure (101,325 Pa), ambient atmospheric temperature (22 °C), ambient atmospheric humidity (50 \%RH)}

Design operates in terrestrial conditions.

Ambient pressure: tank, bladder, and nozzle are designed to produce indicated outputs at standard air pressure
Ambient temperature and humidity: less concern, housing is sealed and insulated

\section{Design Approach and Innovation}
% With a focus on the progress or changes made during Phase 2 of the challenge, outline how the design approaches the problem of food production technology for spaceflight in a novel of innovative manner. (limit 2000 characters)
% More specifically:
% How is your food production technology different from the currently available technology used for food production?
% What are the innovative or novel elements of your food production technology?

\textbf{Control Range and Parameter Independence}

Unlike large-scale vertical farming, PeaPod’s isolation and parameter independence lets it simulate any climate. Wide LED spectrum can emit both near-infrared and near-ultraviolet light, important for creating hormonal responses and compounds in plants. Combined with insulation, humidification and dehumidification, and thermoelectric heating/cooling, PeaPod can generate extreme environments and even conditions on other planets (minus gravity and atmospheric changes). 

These parameters are independent: e.g., lighting heat is countered by thermoregulation cooling. Also, inline nutrient and pH solution injection eliminates drawbacks of reservoir use by taking less space, less solution mixing time, and avoiding a control loop of nutrients/pH (while also being more accurate). 

%These innovations allow PeaPod to grow virtually any plant. 

\textbf{Form Factor and Extendability}

The range of output environments is also possible because of the form factor of a single PeaPod unit. Warehouse-scale vertical farming cannot provide wide-spectrum control due to size---poor insulation and air circulation prohibits extreme temperatures. PeaPod solves this with a small, modular design that enables robust lighting, heating, and cooling systems at home and at scale.

Space savings are a benefit of this feature, as the output of an entire farming or hydroponics setup (requiring a flat field or warehouse) can be spread through unused space (corners, under shelves/desks, etc.) via many small PeaPod setups. This means a large yield can be had without construction, zoning, labour, or any of the other issues accompanying large farming or hydroponic setups.

\textbf{Optimization}

Existing approaches to plant optimization are simple and ineffective, relying on a \textbf{fixed/unchanging} environment parameter set and only examining \textbf{final} plant metrics. This approach is severely limited, in that it does not account for changes over time.

Instead, statistical model is used which takes into account the cumulative property of growth. By monitoring all environment and plant indicators, repeatable and controlled trials with scientific validity are able to be performed. PeaPod counters declining health/quality indicators in real time, and generates tailored programs that to maximize any metric and target further improvement.

% See the preliminary calculations Appendix \ref{app:optimization} for details.

\textbf{Open-Source Design and Data}

Since PeaPod is standard and open-source, units can be had in bulk and assembled by anybody. Public contribution to the project (both in design improvement and data) increases the reliability and safety of the solution. Collected data is, ideally, committed to the public, meaning anybody with a PeaPod can run the same iteration with the same program and species to boost scientific validity, or run a different program to expand PeaPod's knowledge base.

\section{Scientific and Technical Merit}
% With a focus on the progress or changes made during Phase 2 of the challenge, explain how your proposed food production technology demonstrates scientific and technical merit. (limit 4000 characters)
% More specifically:
% Outline your objectives, approaches, and plans you created for developing and verifying the food production technology.
% Please describe your scientific and technical methodology.

\section{Feasibility of Design}
% With a focus on the progress or changes made during Phase 2 of the Challenge, please describe how your technical approach is technically and realistically feasible for use in future space missions (limit 4000 characters).
% More specifically:
% Describe how the related operation processes and procedures are technically and realistically feasible, including (but not limited to):
% Food production
% Food handling
% Maintenance
% Repair
% What alterations would need to be made to make your food production technology feasible in a space flight environment?
% Ensure the emphasis is on future space missions.

% Are there any risks that your food production technology may pose in future space missions, and what are your proposed solutions to mitigate these risks? (limit 2000 characters)

\section{Terrestrial Potential}
% With a focus on the progress or changes made during Phase 2 of the challenge, please outline how the food production technology has the potential to improve terrestrial food production. (limit 10000 characters)
% This may include but is not limited to: Improved cost-benefit, quality and shelf-life of food for harsh and isolated environments, greater and improved food production for food deserts in mild environments and urban centers, new rapid-deployment technologies for humanitarian responses to disasters, and energy used in an efficient way and/or improved process control using e.g. machine learning

% \textbf{Customer-facing Food Service} %OpenComment does this make sense? feels a little weak/unfounded - NV

% A restaurant serving fresh produce needs either a local supplier or a substantial amount of outdoor space. Both of these are cost-prohibitive, and the latter is impossible in many urban situations. Local suppliers' high costs are the result of a few things:
% \begin{itemize}
%     \item Limited seasonal availability
%     \item Frequent transport need
%     \item High costs with little demand
% \end{itemize}
% PeaPod has the potential to reduce these barriers in a cyclic way. Partnerships between local suppliers and restaurants will provide these restaurants with space- and time-efficient PeaPod units with the purpose of generating both produce and data. The increase in produce will reduce the frequency at which suppliers need to make deliveries (reducing cost and spreading demand), while the data generated will let suppliers maximize output, quality, and longevity. Over time, this increases efficiency to the point where local suppliers can provide produce at a lower price, increasing the presence of fresh, local, produce in cities worldwide.

% Another food service application of PeaPod is the globalization of otherwise endemic crops. For example, Wasabi is a difficult-to-grow Japanese crop that is near-impossible to grow overseas (save a few recent commercial developments). PeaPod units in restaurants solve this issue by re-creating the necessary precise conditions and eliminating barriers in skill and time that otherwise impact Wasabi's production and consumption overseas. And, of course, PeaPod's data collection capabilities can be harnessed by farmers to discover more safe, efficient, and adaptable methods of producing crops such as these.

\textbf{Agriculture-as-a-Service}

PeaPod’s modularity and ease of storage can turn unused city spaces to PeaPod farms. With fresh produce in local areas, PeaPod can be a direct food system to paying customers. A subscription service would provide patrons with fresh produce without a middle man. By eliminating transport, distribution, and grocery stores, PeaPod creates fresher, better produce for the general public at a lower cost.

\textbf{Crowd-Sourced Research}

PeaPod's automation is unique in the research space, allowing for autonomous, off-site research. Universities save costs related to space and energy usage by subsidizing PeaPods to individuals, schools, or even restaurants. Users receive sets of parameters to grow crops with, sending data back to the institution and using the produce at the cost of space and energy. The result is a massive dataset from identical conditions in different places, verified by comparison with devices conducting the same tests.

This is an effective tool in climate change famine aid. By predicting conditions in at-risk areas, researchers can conduct tests ahead of time to determine what seeds, traits, and care parameters are most effective for certain conditions. This also informs development of seeds specialized for extreme climates, letting areas counteract food scarcity by having a variety of options prepared ahead of time.

\textbf{De-centralized Production}

Many crops are endemic to certain climates, making global transport necessary to for foreign markets. This reduces freshness, necessitates preservatives, and increases the carbon footprint of agriculture. By upscaling PeaPod technology to a farm scale, climate-bound crops can be produced anywhere. This creates regional farms of global variety, making it easier to have a local food diet.

PeaPod's form factor makes it a viable tool for at-home production, either in cities or off-grid. With only a solar power source, water, and a compact supply of nutrient solutions, users can sustain crops even through winter without travelling for nutrients and supplies.

% \textbf{Blockchain and "PlantCoin" Cryptocurrency Mining}

% In the same way current cryptocurrencies reward nodes for validating all information on a blockchain, a PeaPod network would reward nodes (individual PeaPods) for validating data produced by other nodes. When tens or hundreds of units produce statistically similar results in independent trials using the same parameters, all units are rewarded for their contribution to the network. This process is commonly known as "mining". This allows consumers the option to operate a PeaPod as a passive source of income, with the unit paying for itself by both producing food and mining a "PlantCoin" cryptocurrency.

% This also incentivizes the network to find critical points in the n-dimensional set of data that would otherwise not be checked. For example, few users would intentionally grow plants in drought conditions at this will produce subpar produce. But, by incentivizing users to fill sparse sections of data, we can generate information critical to developing crops for territories affected by climate change.

%we love this idea but idt nasa cares

\textbf{Food Infrastructure Micro-Loans}

For many, finding fresh produce is a struggle whilst growing your own is prohibitively expensive. Micro-loan platforms have attempted to solve this by letting donors fund an interest-free loan for technology/infrastructure which then pays the loan as a percentage of its surplus. 

Unfortunately, these are only feasible for individuals in rural areas with arable land and climate.

PeaPod brings this solution to low-income urban areas with a platform for donors to micro-loan PeaPod units and inputs. The user feeds themselves and sells surplus, while a percentage of sales go to the interest-free loan. Once paid, PeaPod continues to produce food while sales fund its operation. 

This creates permanent, self-sustaining agricultural infrastructure that pays for itself as it grows, requiring little initial capital. This means entire farms throughout high density buildings generating yield with little lost space and almost no labour.

\section{Results}

\subsection{Acceptability of Process}
% Teams must demonstrate the operations processes and procedures, including (but not limited to):
% Operational footprint (i.e., how much space is needed for the solution and its related processes?)
% Food production technology set up
% Food production cycle, including steps to produce food products
% Food handling, processing procedures and collection of food products
% Shutdown, cleaning, maintenance, repair and/or stowage procedure(s)
% An estimate of the overall crew time to operate and maintain the technology
%  (limit 10000 characters):

\textit{Footprint}:

Due to PeaPod's modular construction, the footprint can vary. For a Standard Extended PeaPod, the footprint would be 2m x 0.5m, or about three standing refrigerators. When stowed, volume is reduced to 37\% of when assembled. Control module is packed pre-assembled.

The following estimates are for a Standard Extended PeaPod.

\textit{Setup Process}:

\begin{enumerate}
    \item Assemble housing - \textbf{2 hours}
    \item Install control module(s):
    \begin{enumerate}
        \item Hook up water, power, and network inputs - \textbf{5 min}
        \item Fill nutrient and pH adjustment solution containers - \textbf{10 min}
        \item Mount CM to housing - \textbf{5 min}
    \end{enumerate}
    \item Assemble all trays - \textbf{1 hour},
    \item For each tray, either:
    \begin{enumerate}
        \item Mount lighting boards and driver, daisy chain boards to driver, hook up power and signal to driver and CM - \textbf{10 min} per tray, \textbf{OR}
        \item Mount aeroponic nozzle mount and arm, hook up water delivery line to nozzles and CM - \textbf{10 min} per tray
    \end{enumerate}
    \item UV sterilization - \textbf{20 min}
    \item Prepare and plant seeds for desired crop output, seal growth environment - \textbf{30 min}
    \item Enable primary power supply, and power on automation system, allow to perform self-test and calibrations - \textbf{10 min}
    \item Open water input shutoff valve
    \item Input program for required environments and activate - \textbf{5 min}
\end{enumerate}

Total setup process time (Standard: 2 trays per unit, 12 units, 1 CM): \textbf{8.5 hours} (\textit{one person}) or \textbf{2.5 hours} (\textit{crew of 4})

\newpage

\textit{Food Production Cycle}:

\begin{enumerate}
    \item Environment is maintained, and environment can be observed live at a computer terminal via sensor data and camera feed
    \item Circulation fans enable automated pollination
    \item Perform maintenance, including:
    \begin{itemize}
        \item Cleaning nozzle once a month - \textbf{10 min}
        \item Swapping and recharging dehumidification cartridges when instructed - \textbf{5 min} (active time)
        \item Refilling solution containers when instructed - \textbf{5 min}
    \end{itemize}
    \item Upon End-Of-Program (EOP) notification, the gas exchange system will conduct a "full equalization flush", bringing the internal environment in equilibrium with the surroundings. Users will harvest and store food products (or prepare and consume them immediately, varying time) - \textbf{15 min}
    \item Upon End-Of-Life notification (may occur at the same time as EOP), the plant is scrapped (\textbf{15 min}), and new plants may be planted
\end{enumerate}

Total maintenance time per week: \textbf{1-2 hours} (depending on program)

\textit{Process Evaluation}:

Setup and maintenance processes are fully documented in a "User Manual", which includes both text instructions (with numerical specifications for different actions) as well as diagrams for reference. Notifications from computer refer users to specific subsections of the Manual for maintenance actions. All processes require no specific expertise, just the ability to operate basic hand tools and follow instructions.

All interactions with the automation system (i.e. program upload, environment/camera monitoring) can be accomplished either via a touchscreen panel on the front of the control module or over the Internet.

\subsection{Acceptability of Product}
% Please provide an assessment (using industry standards and existing research) that the food outputs of their technology are likely to meet the acceptability target/rating described. This assessment should include the palatability of the food product (appearance, aroma, texture and flavour) (limit 6000 characters).

There are several considerations when examining the acceptability of the products of our system:

\begin{enumerate}
    \item Produce is not only eaten fresh, but also forms the basis for an innumerable variety of combined and prepared foods (i.e. fresh tomatoes vs. tomato sauce);
    \item When considering prepared derivatives of the food products, the quality of the preparation is a key factor in acceptability. As such, proper care in training and is to be taken;
    \item The products formed by the system (and their properly prepared derivatives) are not exceptional or novel. They are the same plant-based foods grown, consumed, and \textbf{accepted} terrestrially, just grown in a more efficient and controlled way. As such, their acceptability is determined to be of \textbf{equal or greater value};
    \item Plant-environment optimization can be targeted not only at nutritional value or efficiency, but also at acceptability. The feedback can be gathered either through crew Hedonic rating (i.e. tomatoes grown in environment ABC rate X in appearance, Y in aroma, etc.) or more sophisticated analysis (i.e. computer vision analysis of color/size/shape for appearance, tissue concentrations of various aroma/flavor compounds); %TODO: cite some times when this has been performed, i.e. MIT basil study
\end{enumerate}

\section{Safety}

\documentclass{report}
\usepackage{setspace} % Setting line spacing
\usepackage{ulem} % Underline
\usepackage{caption} % Captioning figures
\usepackage{subcaption} % Subfigures
\usepackage{geometry} % Page layout
\usepackage{multicol} % Columned pages
\usepackage{array,etoolbox}
\usepackage{fancyhdr}
\usepackage{enumitem}
\usepackage[toc,page]{appendix}
\setlist{noitemsep}

% Page layout (margins, size, line spacing)
\geometry{letterpaper, left=1in, right=1in, bottom=1in, top=1in}
\setstretch{1}

% Headers
\pagestyle{fancy}
\lhead{PeaPod - Testing Plan}
\rhead{PeaPod Technologies Inc.}

\begin{document}

\begin{titlepage}
    \begin{center}
        \vspace*{1.2cm}

        \textbf{\large{PeaPod - Testing Plan}}

        \vspace{0.5cm}

        NASA/CSA Deep Space Food Challenge Phase 2

        \vfill
        \small{
    \textbf{Jayden Lefebvre - Founder, Lead Engineer}\\
    Port Hope, ON, Canada\\
    \vspace{.5cm}
    \textbf{Nathan Chareunsouk - Design Lead}\\Toronto, ON, Canada\\
    \vspace{.5cm}
    \textbf{Navin Vanderwert - Design Engineer}\\
    BASc Engineering Science (Anticipated 2024), University of Toronto\\
    Toronto, ON, Canada\\
    \vspace{.5cm}
    \textbf{Jonas Marshall - Electronics Engineer}\\
    BASc Computer Engineering (Anticipated 2024), Queen's University\\
    Kingston, ON, Canada
}

\vspace{1cm}

Primary Contact Email: contact@peapodtech.com
        \vspace{.75cm}

        Revision 0.1\\
        PeaPod Technologies Inc.\\
        January 9th, 2022

    \end{center}
\end{titlepage}

\thispagestyle{plain}

\tableofcontents
\newpage

\section{Testing Procedure}
% Per-requirement testing plan
% Processes, measurements, experiments
% NOTE: Often, acceptable ranges in results depend on specifics of the technology

\subsection{Acceptability}
% Are people willing to eat output? Incl. appearance, aroma, flavor, and texture
% Specific preparation and/or incorporation of output (formulation) for tasting

% Suggestion: Double-blind volunteer study?

\subsection{Safety of Process}
% SPECIFICALLY food production area
% Chemical hazards - Toxins, heavy metals (As, Cd, Hg, Pb, etc.)
% Biohazards - total aerobic plate count, ATP testing of food contact surfaces
% TODO: What do those^ mean? How do we test them?

\subsection{Safety of Outputs}
% SPECIFICALLY food outputs
% Biohazards - total aerobic count, specific pathogens (enterobacteriaceae, salmonella, yeasts, molds, E. coli, Listeria, etc.)

\subsection{Resource Outputs}
% Nutritional analysis - macro- and micro-nutrients

\subsection{Reliability and Stability of Outputs}
% Biohazards - total aerobic plate count, enterobacteriaceae, yeasts, molds

\section{Sample Collection Procedure and Schedule}
% Days/cycles of operation before sample collection?
% Incl. sample collection (size, timing, quantity), packaging, shipping

\section{Hazard Analysis and Critical Control Point (HACCP) Plan}
% Def'n of terms: https://www.fda.gov/food/hazard-analysis-critical-control-point-haccp/haccp-principles-application-guidelines#princ
% HACCP reference: https://spinoff.nasa.gov/moon-landing-food-safety?utm_source=TWITTER&utm_medium=KathyLueders&utm_campaign=NASASocial&linkId=141839394

\subsection{Food Production System Description}
% Incl. flowchart


\subsection{Critical Points}
%assembly: only approved parts that have been checked for leaks, contamination, materials, air-tightness

%seeds: vetted and tested breed from an approved supplier, not opened until moment of planting, food safety steps followed while doing so (hand washing, gloves, masks, hairnet...)

%growth medium: tested for bacteria/pests, handled carefully before installation

%system inputs: filtered air, clean water, properly sourced nutrients, all tested by appropriate standards

%maintenance: plants properly isolated when non-food safe materials are present---i.e. if LED boards need to be swapped, if something needs to be greased, if something needs to be glued

%harvesting: all food safety guidelines to be followed - hand washing, gloves, masks, washing, etc.

%re-planting: checking for no old growth medium, sanitizing growth tray with food-safe sanitizer

\subsubsection{Critical Point A}
\textbf{Hazard Description}
% aka hazard analysis, discover CCPs

\textbf{Critical Limits}
% What is the safe range AND the failure conditions for the CCP

\textbf{Monitoring Procedures}
% How do we know the state of the CCP

\textbf{Deviation Procedures}
% aka corrective actions
% How do we keep the CCP within range?

\textbf{Associated Documents}
% record-keeping and documentation procedures

% TODO: Verification Procedures?

\subsubsection{Critical Point ...}

\subsection{Standard Test Record}
% TODO: What exactly is this?

\subsubsection{Purpose and Summary}

\subsubsection{Safety and Quality}

\subsubsection{Test Processes}

% Preparation of Inputs
% Verification
% Setup, Maintenance, and Collection Protocols
% Storage
% Cleanup and Turnover

\subsubsection{Closeout}

\newpage

% References
\bibliographystyle{IEEEtran}
\bibliography{references}
\end{document}

\section{Hazard Analysis and Critical Control Point (HACCP) Plan}
% Def'n of terms: https://www.fda.gov/food/hazard-analysis-critical-control-point-haccp/haccp-principles-application-guidelines#princ
% HACCP reference: https://spinoff.nasa.gov/moon-landing-food-safety?utm_source=TWITTER&utm_medium=KathyLueders&utm_campaign=NASASocial&linkId=141839394

\subsection{Food Production System Description}
% Incl. flowchart


\subsection{Critical Points}
%assembly: only approved parts that have been checked for leaks, contamination, materials, air-tightness

%seeds: vetted and tested breed from an approved supplier, not opened until moment of planting, food safety steps followed while doing so (hand washing, gloves, masks, hairnet...)

%growth medium: tested for bacteria/pests, handled carefully before installation

%system inputs: filtered air, clean water, properly sourced nutrients, all tested by appropriate standards

%maintenance: plants properly isolated when non-food safe materials are present---i.e. if LED boards need to be swapped, if something needs to be greased, if something needs to be glued

%harvesting: all food safety guidelines to be followed - hand washing, gloves, masks, washing, etc.

%re-planting: checking for no old growth medium, sanitizing growth tray with food-safe sanitizer

\subsubsection{Critical Point A}
\textbf{Hazard Description}
% aka hazard analysis, discover CCPs

\textbf{Critical Limits}
% What is the safe range AND the failure conditions for the CCP

\textbf{Monitoring Procedures}
% How do we know the state of the CCP

\textbf{Deviation Procedures}
% aka corrective actions
% How do we keep the CCP within range?

\textbf{Associated Documents}
% record-keeping and documentation procedures

% TODO: Verification Procedures?

\subsubsection{Critical Point ...}

\subsection{Standard Test Record}
% TODO: What exactly is this?

\subsubsection{Purpose and Summary}

\subsubsection{Safety and Quality}

\subsubsection{Test Processes}

% Preparation of Inputs
% Verification
% Setup, Maintenance, and Collection Protocols
% Storage
% Cleanup and Turnover

\subsubsection{Closeout}

% \subsection{Materials}
\subsubsection{System}

\textbf{Automation}\\

\begin{table}[!h]
    \centering
    \begin{tabular}{|c|l|l|l|c|}
    \hline
        Index   & Manufacturer Part Number  & Manufacturer Name         & Description                       & Quantity  \\ \hline
        1       & A000005                   & Arduino                   & ARDUINO NANO ATMEGA328 EVAL BRD   & 1         \\ \hline
        2       & S404GSEJ6-U3000-3         & "Delkin Devices, Inc."    & 4GB MLC MICROSD CARD (-25C - +85  & 1         \\ \hline
        3       & 61304021121               & Würth Elektronik          & CONN HEADER VERT 40POS 2.54MM     & 1         \\ \hline
        4       & SC0510                    & Raspberry Pi              & ZERO 2 W                          & 1         \\ \hline
        5       & DMN2005K-7                & Diodes Incorporated       & MOSFET N-CH 20V 300MA SOT23-3     & 2         \\ \hline
        6       & RC0603FR-0710KL           & YAGEO                     & RES 10K OHM 1\% 1/10W 0603        & 5         \\ \hline
        7       & 4484                      & Adafruit Industries LLC   & MINI PITFT 1.3 FOR RASPBERRY PI   & 1         \\ \hline
        8       & 5055670271                & Molex                     & CONN HEADER SMD R/A 2POS 1.25MM   & 2         \\ \hline
        9       & 5055670471                & Molex                     & CONN HEADER SMD R/A 4POS 1.25MM   & 5         \\ \hline
        10      & 5055670871                & Molex                     & CONN HEADER SMD R/A 8POS 1.25MM   & 3         \\ \hline
        11      & 5055670681                & Molex                     & CONN HEADER SMD R/A 6POS 1.25MM   & 3         \\ \hline
    \end{tabular}
    \caption{Automation system electronic components.}
    \label{tab:automation_components}
\end{table}

In addition, 1x \textit{Automation Motherboard PCB}: 2 Layers, 1 oz. Copper, 1.6mm Thickness, Suggested: HASL Finish (Lead-Free), White PCB, Black Silkscreen

\textbf{Housing}\\


\textbf{Aeroponics}\\


\textbf{Leaf-Zone Thermoregulation}\\


\textbf{Humidification}\\


\textbf{Dehumidification}\\


\textbf{Gas Composition Regulation and Exchange}\\


\textbf{Lighting}\\


\subsubsection{Inputs}
% Supply inputs (water, power, network), consumable inputs (pH/nutrient solutions, dehumidification cartridge)

\textbf{Supply Inputs}
\begin{itemize}
    \item \textit{Water}: reverse-osmosis, ambient
    \item \textit{Power}: 120V 60Hz AC\footnote{The power supply can be altered to suit a variety of power inputs (i.e. DC)}
    \item \textit{Network}: ethernet or wireless, optional
\end{itemize}

\textbf{Consumable Inputs}
\begin{itemize}
    \item \textit{Nutrient/pH Adjusment Solutions}: pouches
    \item \textit{Dehumidification Cartridge}: recharged
\end{itemize}

\subsubsection{Outputs}
% Food outputs(?), by-products/waste (waste water from flushing, dehumidification cartridge)

\textbf{Food Outputs}\\


\textbf{By-Products \& Waste}\\


\subsubsection{Maintenance}

\textbf{Spare Components}\\


\textbf{Tools}\\


\subsubsection{Cleaning}

\textbf{Soaps}\\


\textbf{Disinfectants}\\


\textbf{Tools}\\



% Please outline any operational risks for the technology, and your potential mitigation strategy. (limit 4000 characters)
% More specifically, please outline the following environmental & process safety requirements:
% Avoidance of hazardous compounds or materials used or produced (e.g., microbes, off-gassing, toxic components)
% Avoidance of hazards associated with cleaning this technology prior to and/or after use
% Avoidance of physical, chemical, or biological hazards associated with the hardware or the process
% Clear mitigation strategies to address the aforementioned risks

\section{Inputs and Outputs}
% With a focus on the progress or changes made during Phase 2 of the challenge, please describe the inputs and their amounts needed for your food production technology (limit 6000 characters).

% Inputs could include:
% Chemicals (e.g., nutrients, acid, base, disinfection)
% Water (RO, tap etc.)
% Energy
% Cleaning supplies
% CO2 or other gases
% Other materials that enter the system
% NOTE: Unlike in Phase 1, it is no longer acceptable to estimate the quantities of your inputs based on reasonable literature values. Instead, Teams must provide data generated from direct measurement of their system, or calculations based on known system metrics (i.e. the basis for calculations must be gathered from the prototype itself, and not hypothetical).

\textit{Infrastructural Inputs}: Reverse osmosis water (constant supply at positive pressure from onboard RODI system), nutrient solutions (stored, one container each plus refill tanks), pH solutions (one container pH up, one container pH down, plus refill tanks, stored), power (onboard power, standard 120V AC 60Hz), network connection (onboard network, for remote control, live video/data transmission), plant seeds (stored in vacuum-sealed seed bank, selected for variety and acceptability), input air (HEPA filtered, carbon dioxide-rich)

\textit{Process Inputs}: Plant species identifiers, environment program (for entire growth cycle, one per plant species), nutrient and pH-adjustment solution identifiers (compounds and molarities, i.e. solution A is 0.6M NaNO${}_3$)

Common nutrient solutions target specific ions, including bioavailable nonmetals (nitrates/nitrites, ammonia/ammonium, phosphates, sulphates), metals (potassium, calcium, magnesium, iron) and other trace elements.

% With a focus on the progress made during Phase 2 of the challenge, describe the outputs of your food production technology system. (limit 3000 Characters)

% Outputs could include:
% Food products
% Waste (wastewater, inedible biomass, cleaning wipes, testing material etc.)
% Heat (latent and sensible)
% Other useable or unusable product exiting the system, including liquid and gaseous process flows (e.g., water vapor, low-molecular weight organic and inorganic compounds, water, oils, etc.)

\textit{Products}: Edible plant matter, recorded environment data, plant metric data, live video feed, time-lapse capture

\textit{Byproducts}: Inedible plant matter (stems/roots/leaves/etc., waste), sensible heat (from thermoregulation pumping, managed by onboard heating/cooling), exhaust air (via HEPA filter, sterilized and dehumidified by onboard life support, oxygen-rich), minimal water vapour (as a result of higher air humidity, minimized by housing seal), latent heat (as a result of higher leaf zone temperature, minimized by insulation)

% With a focus on the progress made during Phase 2 of the challenge, please outline the net water consumption of your food production technology (limit 4000 characters).

% Note: As your team’s net water consumption is previously captured, you may copy and paste your response from the constraint section above.

% With a focus on the progress or changes made during Phase 2 of the challenge, describe how the food production technology achieves the greatest amount of food output in relation to the quantity of inputs and quantity of waste output (limit 4000 characters)

\section{Reliability and Stability}
% Reliability: Operational cycle time (i.e., amount of time for one production cycle) (limit 2000 characters)

% Reliability: System lifespan (i.e., how many cycles can the system perform before requiring replacement parts?) (limit 2000 characters)

% Reliability: Overview of maintenance/repair as well as cleaning processes and procedures, including (but not limited to): (limit 8000 characters)
% What the processes and procedures look like
% Maintenance schedule (i.e., how often will it need maintenance and/or cleaned?)
% Which component(s)/element(s) require cleaning 
% Amount of time required to perform the system cleaning and maintenance
% What is needed to fully clean the system (e.g., water, chemicals)
% Which component(s)/element(s) require maintenance or replacement (i.e., what components will need to be replaced, and when?)
% Critical spare parts for a three-year mission and longevity of those spare parts
% Estimated additional mass needed for replacement parts and cleaning materials
% Estimated total time needed per month, per 6 months, and per year for maintenance and cleaning

% Stability: Demonstrate the stability of both the input products used and food product outputs and provide rationalization of the estimated time the inputs and outputs will be fit for use and/or consumption (i.e., shelf-life). (limit 8000 characters)

\subsection{Materials}
\subsubsection{System}

\textbf{Automation}\\

\begin{table}[!h]
    \centering
    \begin{tabular}{|c|l|l|l|c|}
    \hline
        Index   & Manufacturer Part Number  & Manufacturer Name         & Description                       & Quantity  \\ \hline
        1       & A000005                   & Arduino                   & ARDUINO NANO ATMEGA328 EVAL BRD   & 1         \\ \hline
        2       & S404GSEJ6-U3000-3         & "Delkin Devices, Inc."    & 4GB MLC MICROSD CARD (-25C - +85  & 1         \\ \hline
        3       & 61304021121               & Würth Elektronik          & CONN HEADER VERT 40POS 2.54MM     & 1         \\ \hline
        4       & SC0510                    & Raspberry Pi              & ZERO 2 W                          & 1         \\ \hline
        5       & DMN2005K-7                & Diodes Incorporated       & MOSFET N-CH 20V 300MA SOT23-3     & 2         \\ \hline
        6       & RC0603FR-0710KL           & YAGEO                     & RES 10K OHM 1\% 1/10W 0603        & 5         \\ \hline
        7       & 4484                      & Adafruit Industries LLC   & MINI PITFT 1.3 FOR RASPBERRY PI   & 1         \\ \hline
        8       & 5055670271                & Molex                     & CONN HEADER SMD R/A 2POS 1.25MM   & 2         \\ \hline
        9       & 5055670471                & Molex                     & CONN HEADER SMD R/A 4POS 1.25MM   & 5         \\ \hline
        10      & 5055670871                & Molex                     & CONN HEADER SMD R/A 8POS 1.25MM   & 3         \\ \hline
        11      & 5055670681                & Molex                     & CONN HEADER SMD R/A 6POS 1.25MM   & 3         \\ \hline
    \end{tabular}
    \caption{Automation system electronic components.}
    \label{tab:automation_components}
\end{table}

In addition, 1x \textit{Automation Motherboard PCB}: 2 Layers, 1 oz. Copper, 1.6mm Thickness, Suggested: HASL Finish (Lead-Free), White PCB, Black Silkscreen

\textbf{Housing}\\


\textbf{Aeroponics}\\


\textbf{Leaf-Zone Thermoregulation}\\


\textbf{Humidification}\\


\textbf{Dehumidification}\\


\textbf{Gas Composition Regulation and Exchange}\\


\textbf{Lighting}\\


\subsubsection{Inputs}
% Supply inputs (water, power, network), consumable inputs (pH/nutrient solutions, dehumidification cartridge)

\textbf{Supply Inputs}
\begin{itemize}
    \item \textit{Water}: reverse-osmosis, ambient
    \item \textit{Power}: 120V 60Hz AC\footnote{The power supply can be altered to suit a variety of power inputs (i.e. DC)}
    \item \textit{Network}: ethernet or wireless, optional
\end{itemize}

\textbf{Consumable Inputs}
\begin{itemize}
    \item \textit{Nutrient/pH Adjusment Solutions}: pouches
    \item \textit{Dehumidification Cartridge}: recharged
\end{itemize}

\subsubsection{Outputs}
% Food outputs(?), by-products/waste (waste water from flushing, dehumidification cartridge)

\textbf{Food Outputs}\\


\textbf{By-Products \& Waste}\\


\subsubsection{Maintenance}

\textbf{Spare Components}\\


\textbf{Tools}\\


\subsubsection{Cleaning}

\textbf{Soaps}\\


\textbf{Disinfectants}\\


\textbf{Tools}\\



\newpage

% References
\bibliographystyle{IEEEtran}
\bibliography{references}
\end{document}