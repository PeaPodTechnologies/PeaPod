\documentclass{../tex/report}
\usepackage{setspace} % Setting line spacing
\usepackage{ulem} % Underline
\usepackage{caption} % Captioning figures
\usepackage{subcaption} % Subfigures
\usepackage{geometry} % Page layout
\usepackage{multicol} % Columned pages
\usepackage{array,etoolbox}
\usepackage{fancyhdr}
\usepackage{enumitem}
\usepackage[toc,page]{appendix}
\setlist{noitemsep}

% Page layout (margins, size, line spacing)
\geometry{letterpaper, left=1in, right=1in, bottom=1in, top=1in}
\setstretch{1}

% Headers
\pagestyle{fancy}
\lhead{PeaPod - Solution Overview \& Progress Report}
\rhead{PeaPod Technologies Inc.}

\begin{document}

\begin{titlepage}
    \begin{center}
        \vspace*{1.2cm}

        \textbf{\large{PeaPod - Solution Overview \& Progress Report}}

        \vspace{0.5cm}

        NASA/CSA Deep Space Food Challenge Phase 2

        \vfill
        \small{
    \textbf{Jayden Lefebvre - Founder, Lead Engineer}\\
    Port Hope, ON, Canada\\
    \vspace{.5cm}
    \textbf{Nathan Chareunsouk - Design Lead}\\Toronto, ON, Canada\\
    \vspace{.5cm}
    \textbf{Navin Vanderwert - Design Engineer}\\
    BASc Engineering Science (Anticipated 2024), University of Toronto\\
    Toronto, ON, Canada\\
    \vspace{.5cm}
    \textbf{Jonas Marshall - Electronics Engineer}\\
    BASc Computer Engineering (Anticipated 2024), Queen's University\\
    Kingston, ON, Canada
}

\vspace{1cm}

Primary Contact Email: contact@peapodtech.com
        \vspace{.75cm}

        Revision 1.1\\
        PeaPod Technologies Inc.\\
        August 16th, 2022

    \end{center}
\end{titlepage}

\thispagestyle{plain}

\tableofcontents
\clearpage

\section{Design Status}
% SPECIFICALLY the design process - from establishment of base requirements, to selection of specific materials

\subsection{Completion}

The design is 85\% complete as of May 31st, 2022. All high-level design is complete, and most of what remains is to select specific components for a few subsystems. A highly-accurate 3D model was created to select and validate placement, orientation, and fit of all components prior to prototyping.

\subsection{Process Description}
% Please give a brief update of your food production system’s process (including maintenance and cleaning):
% Fully-detailed process description (incl. setup, operation, maintenance, and cleaning), flow diagram for each
% maximum 5000 characters
\subsubsection{Setup}
\subsubsection{Operation}
\subsubsection{Maintenance}
\subsubsection{Cleaning}

\clearpage

\section{System-Level Report}
% To be extracted as a separate PDF and submitted as 1.3

\subsection{Automation}
\label{sec:automation}

\textbf{Purpose}: Performing growth-, maintenance-, and data-related tasks autonomously on the basis of both schedule and necessity to reduce crew maintenance time, improve consistency of products, and eliminate safety risks. Maintains the homogeneity of the internal environment with increased accuracy and precision over crew interference, while enabling simultaneous control over all parameters.

\textbf{Function}:
\begin{itemize}
    \item \textbf{Inputs}: Environment data stream (sensor readings), growth program
    \item \textbf{Outputs}: Actuator control signals, crew/cloud messaging, environment data (stored), timelapse photo set
\end{itemize}

\textbf{Method}:
\begin{enumerate}
    \item \textit{Setup}:
    \begin{enumerate}
        \item Power is connected and system is booted;
        \item Program is selected by user;
    \end{enumerate}
    \item \textit{Testing}:
    \begin{itemize}
        \item Power-on Self-Test (POST) passes (i.e. all hardware is online and communicating as expected);
        \item Systems enact program as intended (i.e. control systems respond properly);
    \end{itemize}
    \item \textit{Process}:
    \begin{enumerate}
        \item Checks operating preconditions (self POST and per-subsystem);
        \item \textbf{Environment Control Loop} (matches \textit{Sense-Plan-Act} model of robotics): %TODO cite the model
        \begin{enumerate}
            \item \textit{Sense}: Receives and stores data about current environment state;
            \item \textit{Plan}: Compares current state to "desired"/program state, develops a "plan"/actuator control to reach desired state;
            \item \textit{Act}: Controls subsystem operations in order to enact the plan;
        \end{enumerate}
        \item Notifies user on maintenance requirement (i.e. non-automated input/output management, refills, repairs, etc.), end-of-program (EOP), and diagnostic info;
    \end{enumerate}
    \item \textit{Shutdown} (either manual or EOP):
    \begin{enumerate}
        \item Stop all subsystem operations;
        \item Power down;
    \end{enumerate}
\end{enumerate}

\textbf{Features}:

A dual-computer system was chosen, as this allowed for discrete management of high-level functionality (camera capture, internet/cloud functionality, local storage, complex calculations, etc.) and low-level functionality (hardware-level communications, actuator control and GPIO) in a master/slave topology with a constant two-way stream of shared data.

\begin{itemize}
    \item \textit{Computer (Master)}: Manages data collection, batch storage, analysis, and transmission/receiving, as well as planning/calculations for actuator control. Includes internal clock (for program, notification), network connection (for data transmission, notification), photo capture, and non-volatile storage (for data/photos). Sends instructions derived from the program to the microcontroller.
    \item \textit{Microcontroller (Slave)}: Manages detailed sensor/actuator states and communications with them (on/off, sensor readings, actuator control). Streams collected sensor readings to the computer.
    \item \textit{Program}: Set of actuated instructions (e.g. lights on) and control targets (e.g. hold air temperature at 22°C) to enact at specific points in the growth cycle, as well as config data;
    \item \textit{Camera Capture \& Plant Performance Metric (PPM) Extraction}: Top-down and side-view cameras, captured under standard lighting at regular intervals throughout the course of the growth cycle. For live feed transmission to users (local and remote), as well as PPM extraction via computer vision and machine learning for yield and diagnostic analysis. Potentially relevant PPMs include (but are not limited to):
    \begin{itemize}
        \item Leaf health indicators (i.e. leaf tip burn, leaf curl, chlorosis);
        \item Leaf count, size distribution;
        \item Leaf density;
        \item Canopy dimensions/surface area;
        \item Plant height;
        \item Fruit/harvest body size, ripeness;
    \end{itemize}
    \item \textit{Environment Data}: Record the environment's current state. Covers each \textit{control loop} environment parameter (e.g. included in a feedback loop). Control loop environment parameters include:
    \begin{itemize}
        \item Leaf-zone temperature (see \ref{sec:airthermoregulation});
        \item Leaf-zone humidity (see \ref{sec:humidityregulation});
        \item Root-zone temperature (see \ref{sec:aeroponics});
        \item Gas concentrations (see \ref{sec:gas});
    \end{itemize}
    \item \textit{Actuator Control}: Induces change in environment parameters. Covers both \textit{control loop} and \textit{actuated instruction} environment parameters. Actuated instruction environment parameters include:
    \begin{itemize}
        \item Lighting (see \ref{sec:lighting});
        \item Water delivery (see \ref{sec:aeroponics});
        \item Plant nutrient delivery (see \ref{sec:aeroponics});
        \item Water pH (see \ref{sec:aeroponics});
        \item Air circulation rate (see \ref{sec:airthermoregulation});
    \end{itemize}
    \item \textit{Diagnostic Systems}: Include informative sensors tracking system input availability, subsystem diagnostics, etc. as well as notification triggers.
\end{itemize}

\clearpage

\subsection{Housing}
\label{sec:housing}

\textbf{Purpose}: \textit{Isolates} and \textit{insulates} growth environment from surroundings (heat, light, water vapour, air). Provides structural integrity and mounting points for other subsystems. Enables system extendability via repeated "unit cell" topology.

\textbf{Method}:
\begin{enumerate}
    \item \textit{Setup}:
    \begin{enumerate}
        \item Construct frame and install panels;
        \item Mount control module (w/ subsystems), connect inputs and internal subsystem connections;
        \item Install tray mounts, insert trays (w/ subsystems);
    \end{enumerate}
    \item \textit{Testing}:
    \begin{itemize}
        \item Frame construction is rigid, level, and sturdy;
        \item Panels are insulating against temperature changes;
    \end{itemize}
    \item \textit{Process}:
    \begin{enumerate}
        \item Panels insulate against heat gain/loss, are opaque, and contain light and heat via reflection;
        \item Shell construction is tight, thus sealing against moisture;
        \item Internal vertical mounting channels for systems, horizontal plane "trays";
        \item \textbf{Extension} (can be repeated):
        \begin{enumerate}
            \item Add a second housing;
            \item Remove dividing panel from both housings;
            \item Remove "shared" skeleton extrusions from second housing;
            \item Join the two housings to form one larger 2x1 housing;
            \item \textbf{Extension Modes} (may be combined in any way to suit application):
            \begin{itemize}
                \item \textit{Option 1} (Smaller Housings): Operate the combined housing off \textbf{one} control module.
                \item \textit{Option 2} (Larger Housings): Add control modules to account for additional air volume, plant count, power requirement, etc.. Operate in a \textbf{controller-follower topology}.
                \item \textit{Option 3} (Frame Connection Only): Leave the dividing panel, add a control module, and operate the two PeaPods \textbf{separately}.
            \end{itemize}
        \end{enumerate}
    \end{enumerate}
    \item \textit{Shutdown}:
    \begin{enumerate}
        \item Dismount all systems, remove trays;
        \item Disassemble housing;
    \end{enumerate}
\end{enumerate}

\textbf{Features}:
\begin{itemize}
    \item \textit{Frame}: T-slotted aluminum extrusion framing with aluminum face-mounted brackets forms a cubic skeleton for rigidity/strength (high strength-to-weight aluminum) and easy component mounting and repositioning (standard mounting channels). These extrusions form the "edges" of the cubic housing. %Todo: cite extrusion
    \item \textit{Panels}: Graphite-enhanced expanded polystyrene (GPS) rigid foam insulation panels \cite{insulation} with reflective mylar internal lamination increase energy efficiency (GPS RSI of 0.0328$\frac{m^2 \cdot \degree C}{W}$ per mm of thickness, mylar enables light/heat reflection), as well as safety against cross-contamination and pathogens. Panels slide into extrusion channels and form a "seal" for greater water vapour retention. These panels form the "faces" of cube. %Todo: cite mylar
    \item \textit{Trays}: Horizontal plane subframes mounted to internal vertical extrusion channels for ease of repositioning. Trays slide in/out on permanent mounts. All connections are quick-connect (i.e. quick-diconnect tubing for grow tray, push connectors for lighting) for ease of removal. Trays include:
    \begin{itemize}
        \item \textit{Grow Trays}: Support plants (via grow cups), aeroponic nozzles, aeroponics container, and supply/recycling lines (See \ref{sec:aeroponics}).
        \item \textit{Lighting Trays}: Support LED boards, driver board (See \ref{sec:lighting}).
    \end{itemize}
\end{itemize}

\clearpage

\subsection{Aeroponics}
\label{sec:aeroponics}

\textbf{Purpose}: Delivers plant nutrients and pH- and temperature-controlled water to the roots via a fine mist.

\textbf{Function}:
\begin{itemize}
    \item \textbf{Inputs}: Reverse osmosis water\footnote{RO water has no dissolved nutrients and a neutral pH of 7.0. This enables easier and more reliable calculations. In addition, it has no particulate or minerals, minimizing the chances of nozzle clog.} under positive pressure, concentrated pH up \& down solutions and nutrient solutions, nozzle delivery on/off control (\ref{sec:automation}), pH and nutrient solution ratios as control signals (dosing pump speeds; \ref{sec:automation}), water thermoregulation control signal (\ref{sec:automation})
    \item \textbf{Outputs}: pH- and nutrient-controlled water mist (50 micron mean droplet diameter)
\end{itemize}

\textbf{Method}:
\begin{enumerate}
    \item \textit{Setup}:
    \begin{enumerate}
        \item Hook up water, solution, and signal inputs;
        \item Connect the quick-disconnect fitting;
        \item Calibrate pressure, temperature sensors to atmospheric;
        \item Enable water input to prime system (if known pressure/temperature, calibrate sensors);
        \item Mount container, connect runoff collection line to recycling port;
    \end{enumerate}
\newpage
    \item \textit{Testing}:
    \begin{itemize}
        \item Temperature, pressure sensors communicate as expected;
        \item No leaks at any connections under a) source pressure, b) fully pressurized;
        \item Pump actuates and auto-shuts off as expected, and is able to deliver the required pressure;
        \item All components, tubing, and connectors/fittings withstand full pressurization;
        \item Solenoid is normally closed, withstands full pressurization, and opens when power is applied;
        \item Quick-disconnect operates as intended at full pressurization without leaks;
        \item Nozzles produce even-distribution full-cone mist;
        \item Manual and actuated valves operate as intended;
        \item Runoff container is sealed, and runoff collection operates as intended;
    \end{itemize}
    \item \textit{Process}:
    \begin{enumerate}
        \item Water is pressurized to constant 80psi;
        \item Heat is added to or removed from the water;
        \item Temperature and pressure of the water is read (feedback);
        \item Nutrient and pH solutions are mixed in-line at an adjustable ratio\footnote{I.e. add X mL of nutrient solution Y per mL water to achieve Z ppm, or add A mL of pH down solution per mL water to achieve a pH of B.};
        \item Flow to nozzle is controlled (on/off);
        \item Nozzle turns pressurized water into mist;
        \item Runoff is contained by a water-tight container, and collected for recycling;
    \end{enumerate}
    \item \textit{Shutdown}:
    \begin{enumerate}
        \item Power down the pump and thermoregulation unit;
        \item Close the nutrient and pH solution valves;
        \item Close the source shutoff valve;
        \item Open the drain valve, and allow the system to depressurize completely;
        \item Re-open the source shutoff valve and flush the system with fresh water;
        \item Power down the solenoid;
        \item Collect all remaining runoff;
        \item Disconnect the quick-disconnect fitting;
        \item Disconnect the inputs;
    \end{enumerate}
\end{enumerate}

\textbf{Features} (in order of plumbing; source $\to$ nozzle):
\begin{itemize}
    \item \textit{Water Source}: Input for ambient reverse-osmosis water.
    \item \textit{Manual Source Shutoff Valve}: Ball valve.
    \item \textit{Diaphragm Pump}: Self-priming, auto-shutoff at 80psi. Power is controlled by a relay.
    \item \textit{Inline Thermoelectric Water Heater/Cooler Block}: Aluminum water block heat pump. See Section \ref{sec:airthermoregulation}.
    \item \textit{Solution Injection Manifold}: A manifold of parallel inline injectors, allowing for on-demand adjustment of mixing ratios for nutrient and pH solutions. Comprises:
    \begin{itemize}
        \item \textit{Manifold}: Splits the water line into a set of parallel branches with inline tees to enable solution injection.
        \item \textit{Dosing Pumps}: Stepper-motor driven custom peristaltic pumps deliver solutions at a controlled rate/ratio (one per solution). Toleranced to prevent backflow at pressure. % TODO more details (tubing type, washers, bearings, etc.) w/ part numbers
        \item \textit{Nutrient Solutions}: Aqueous. Highly concentrated. Selectable as part of the program (\ref{sec:automation})\footnote{Many different solutions can be combined (according to solubility laws, pH requirements, etc.).}, and may include any of:
        \begin{itemize}
            \item Bioavailable nonmetals (ammonia, ammonium, nitrates, nitrites, phosphates, sulfates, etc.)
            \item Bioavailable metals (potassium, etc.)
            \item Minerals (magnesium, calcium)
            \item Other trace elements
            \item Custom solutions (i.e. fungicides/algicides, descaling solutions)
        \end{itemize} 
        \item \textit{pH Adjustment Solutions}\footnote{\textit{NOTE:} Ionic composition of pH solutions should be considered in the understanding of the nutrient composition (i.e. phosphic acid results in phosphate ions in spray)}: Aqueous. Highly concentrated. One for pH up (>8), one for pH down (<6).
        \item \textit{Solution Storage Containers}: Opaque, insulated, chemical-safe, refillable cartridges. Prevent degradation of solution compounds over time via light or heat.
        \begin{itemize}
            \item \textit{Fill Level Sensors}: Depth sensors measure fill level of container. Notifies user to refill.
        \end{itemize}
    \end{itemize}
    \item \textit{Water Temperature Sensor}: Tee-fitted. Informs a \textbf{PID control loop}. See Section \ref{sec:airthermoregulation}.
    \item \textit{Accumulator Tank}: Uses an air bladder to maintain and stabilize pressure.
    \item \textit{Pressure Sensor}: Allows for shutoff of pump in case of emergency.
    \item \textit{Drain Valve}: Tee-fitted ball valve. Allows the system to be depressurized and drained.
    \item \textit{Solenoid Valve}: Controls delivery to the nozzles to enable on-demand misting.
    \item \textit{Grow Tray Quick-Disconnect}: Connectors between aeroponics supply and nozzles that allow for quick disconnection with auto-shutoff so the trays may be removed.
    \item \textit{Nozzle}: Mounted to grow tray, pointed at plant roots. 80psi water through a 0.4-0.6mm orifice produces 5-50 micron water droplets, optimal for plant growth. This method is 98\% more water-efficient than traditional farming.%TODO: Sources??
    \item \textit{Root-Zone Container}: Watertight container that encapsulates the entire root zone. Made of a woven waterproof composite fabric (CT5K.18 mylar with Dyneema, 1.43oz/yd${}^2$), chosen for high strength-to-weight ratio (15x that of steel) and natural no-coating food-safe waterproof quality \cite{dyneema}. Mounted and \textbf{sealed} to the grow tray with a drawstring for easy root zone access. Provides water supply and runoff collection ports.
\end{itemize}

\clearpage

\subsection{Air Thermoregulation}
\label{sec:airthermoregulation}

\textbf{Purpose}: Maintaining desired leaf-zone air temperature and circulating air.

\textbf{Function}:
\begin{itemize}
    \item \textbf{Inputs}: Power, air temperature control signal (\ref{sec:automation}), air circulation control signal (\ref{sec:automation});
    \item \textbf{Outputs}: $\pm$Heat to environment, $\mp$heat to surroundings, internal air circulation, internal air temperature sensor signal (\ref{sec:automation});
\end{itemize}

\textbf{Method}:
\begin{enumerate}
    \item \textit{Testing}:
    \begin{itemize}
        \item Heat pump direction and magnitude respond to control signal as expected;
        \item Fans operate as expected;
        \item Heat pump power exceeds maximum heat loss (temperature extremes)\footnote{i.e. if X Watts leave the system at MAX$\degree$C internal, and Y Watts enter the system at MIN$\degree$C internal, the heat pump must transfer >X, >Y Watts.};
        \item Heat pump power exceeds that required to reach temperature extremes in under 120 seconds given the system's heat capacity;
    \end{itemize}
    \item \textit{Process}:
    \begin{enumerate}
        \item Air is circulated throughout the environment;
        \item Temperature is measured, sent to control module;
        \item Control module controls heat pump speed and direction (heating vs. cooling);
    \end{enumerate}
\end{enumerate}

\textbf{Calculations}:

Assuming an atmospheric pressure $P$ of 101.325kPa, a surroundings temperature range $T_{surr}$ of 22$\degree$C, a system target temperature range $T_{sys-min}$, $T_{sys-max}$ of 10-35$\degree$C, a molar mass of dry air\footnote{Water vapour has a maximum concentration of 30g/kg at 30$\degree$C, or 3\%, which is negligible for mass and heat capacity calculations.} $M$ of 28.97 $\frac g{mol}$, a specific heat capacity of dry air $c_p$ of $1.006 \frac{J}{g*\text{K}}$, a 4-unit (2x2 units, 16 faces) expanded configuration, and a face insulation RSI per mm of $0.0328\text{m}^2~  \degree \text{C}~\text{W}^{-1}~\text{mm}^{-1}$ (See Section \ref{sec:housing}):\\
\vspace{.05cm}
\begin{gather*}
    \label{eqn:heatloss}
    Q_{loss}=\frac{(T_{surr}-T_{sys-max}) * A}{\text{RSI per mm} * \ell}=\frac{(22\degree \text{C}-35\degree \text{C}) * (16 \text{ faces} * 0.5\text{m} * 0.5\text{m})}{0.0328 \text{m}^2~  \degree \text{C}~\text{W}^{-1}~\text{mm}^{-1} * 25.4 \text{mm}}=-62.42 W\\\vspace{.05cm}\\
    \label{eqn:heatgain}
    Q_{gain}=\frac{(T_{surr}-T_{sys-min}) * A}{\text{RSI per mm} * \ell}=\frac{(22\degree \text{C}-10\degree \text{C}) * (16 \text{ faces} * 0.5\text{m} * 0.5\text{m})}{0.0328 \text{m}^2~  \degree \text{C}~\text{W}^{-1}~\text{mm}^{-1} * 25.4 \text{mm}}=57.61 W\\\vspace{.05cm}\\
    \label{eqn:airmass}
    m_{air}=\frac{P*V*M}{R*T_{avg}}=\frac{101325\text{Pa}*(0.5\text{m}*0.5\text{m}*0.5\text{m}*4\text{ units})*28.97\frac g{mol}}{8.314\frac{J}{\text{mol}*K}*300\text{K}}=588.4g\\
    \vspace{1.5cm}\\
    \text{Continued on next page}
\end{gather*}
\vspace{1cm}

\begin{gather*}
  \label{eqn:heating}
  W_{heating}=\frac{m*c_p*(T_{surr}-T_{sys-max})}{t}=\frac{588.4g*1.006\frac{J}{g*\text{K}}*(22\degree \text{C}-35\degree \text{C})}{120\text{ sec}}=-64.13\text{W}\\
  \label{eqn:cooling}
  W_{cooling}=\frac{m*c_p*(T_{surr}-T_{sys-min})}{t}=\frac{588.4g*1.006\frac{J}{g*\text{K}}*(22\degree \text{C}-10\degree \text{C})}{120\text{ sec}}=59.19\text{W}
\end{gather*}

$\therefore$ A thermoelectric system able to transfer at least 70W (such as \cite{peltier}, which transfers up to 85W) will supply enough power to heat/cool the system from ambient to extremes in 120 seconds and maintain temperature.

\begin{gather*}
  \label{eqn:thermalresistance-hot}
  R_{\theta~Peltier-Surr}=R_{\theta~Peltier-Sink}+R_{\theta~Sink-Air}\le\frac{T_{h~max} - T_{surr}}{Q_{max}}=\frac{50\degree C - 22\degree C}{85W}=0.329\degree \text{C W}^{-1}\\
  \label{eqn:thermalresistance-cold}
  R_{\theta~Peltier-Sys}=R_{\theta~Peltier-Sink}+R_{\theta~Sink-Air}
\end{gather*}

\textbf{Features}:
\begin{itemize}
    \item \textit{Circulation Fans}: Located in growth environment to circulate air even temperature distribution.
    \item \textit{Temperature Sensors}: SHT31 \cite{sht31} sensors on breakout boards located throughout the growth environment to measure air temperature. Informs a \textbf{PID control loop} (\ref{sec:automation}).
    \item \textit{Heat Pump}: Pumps heat in or out of the growth environment. Is comprised of:
    \begin{itemize}
        \item \textit{Peltier Device}: 85W bidirectional solid-state \textbf{thermoelectric device} (aka Peltier tile) \cite{peltier} pumps heat from one face to the other. Better space efficiency, less complexity (no liquids, pressurized fluids, etc.), and more precise than other methods.
        \item \textit{Peltier Driver Circuit}: Controls magnitude and direction of Peltier device heat pump via a \textbf{dimmable voltage source} and \textbf{MOSFET H-bridge}, respectively. See Figure \ref{fig:peltierdriver}.
        \item \textit{Heat Sinks}: Aluminum blocks with fins hold and exchange heat between air and Peltier devices. One set on each side of the Peltier (inside and outside environment) builds "heat pump". Mating face coated with thermal compound for better transfer.
        \item \textit{Heat Sink Fans}: Located on both sets of heat sinks for better heat dissipation.
    \end{itemize}
\end{itemize}

\begin{figure}[h]
  \centering
  \includegraphics[width=0.8\textwidth]{images/thermosim.png}
  \hfill
  \caption{Peltier driver circuit simulation (live version: \cite{thermo-falstad})}
  \label{fig:peltierdriver}
\end{figure}

\clearpage

\subsection{Leaf-Zone Humidity Regulation}
\label{sec:humidityregulation}

\textbf{Purpose}: Regulates the relative humidity of the leaf zone.

\textbf{Function}:
\begin{itemize}
    \item \textbf{Inputs}: Humidification on/off control signal (\ref{sec:automation}), dehumidification on/off control signal (\ref{sec:automation})
    \item \textbf{Outputs}: Humidification or dehumidification on-demand
\end{itemize}

\textbf{Method}:
\begin{enumerate}
    \item \textit{Process}:
    \begin{enumerate}
        \item When humidity is too low (below dead-zone), humidification is activated;
        \item When humidity is too high (above dead-zone), dehumidification is activated;
        \item When humidity is at target (within dead-zone), both systems are deactivated;
    \end{enumerate}
\end{enumerate}

\textbf{Features}:
\begin{itemize}
    \item \textit{Humidification System}: See Section \ref{sec:humidification}.
    \item \textit{Dehumidification System}: See Section \ref{sec:dehumidification}.
    \item \textit{Humidity Sensors}: Multiple temperature and humidity sensors \cite{sht31} on small daughterboards frame-mounted throughout the growth environment to measure air relative humidity (\%RH). Informs the \textbf{bang-bang control loop}.
    \item \textit{Bang-Bang Control Loop}: A bang-bang (on/off) control loop with a hysteresis dead-zone (see equation \ref{eqn:bangbang}). Humidity sensors inform the loop, "error" is calculated (current vs desired humidity), and this informs whether or not to activate either the humidification or dehumidification systems (or neither). Requires tuning of dead-zone (automatic). Built into the automation system (see \ref{sec:automation});
\end{itemize}

\begin{gather}
    \label{eqn:bangbang}
    u(t)=\begin{dcases}
        -1  &   x < -d \\
        0   &   -d \leq x \leq d \\
        1   &   x > d \\
    \end{dcases}
\end{gather}

\clearpage

\subsubsection{Humidification}
\label{sec:humidification}

\textbf{Purpose}: Actively \textit{increases} growth environment air humidity.

\textbf{Function}:
\begin{itemize}
    \item \textbf{Inputs}: Power, humidification on/off control signal (\ref{sec:automation}), RO water\footnote{RO water contains no minerals/particulate, and as such prevents the common problem of mesh clog/calcification.}
    \item \textbf{Outputs}: Dry water vapour (\ref{sec:automation})
\end{itemize}

\textbf{Method}:
\begin{enumerate}
    \item \textit{Setup}:
    \begin{enumerate}
        \item Connect humidification control signal to control module.
        \item Connect RO water line to water tank.
    \end{enumerate}
    \item \textit{Testing}:
    \begin{itemize}
        \item Humidification unit responds to control signal as expected;
        \item Humidity sensor reads as expected;
        \item Tank does not leak;
    \end{itemize}
    \item \textit{Process}:
    \begin{enumerate}
        \item Water is delivered to a small tank (nebulizer is mounted);
        \item Power and control signal activate a nebulizer driver;
        \item Nebulizer vapourizes water;
    \end{enumerate}
    \item \textit{Shutdown}:
    \begin{enumerate}
        \item Disconnect RO water line and drain tank;
        \item Disconnect control signals from control module;
    \end{enumerate}
\end{enumerate}

\textbf{Features}:
\begin{itemize}
    \item \textit{Circulation Fans}: To circulate dry water vapour for even humidification. See Section \ref{sec:airthermoregulation}.
    \item \textit{Humidification Unit}: Easily controllable and produces a consistent vapour. Comprised of:
    \begin{itemize}
        \item \textit{Water Tank}: Holds a small amount of water behind the piezoelectric mesh.
        \item \textit{Mesh Nebulizer}: Piezoelectric ceramic disc with a microporous stainless steel mesh in the center. Oscillates in such a way that dry vapour is generated when water is passed over the mesh. Mounted to the water tank.
        \item \textit{Driver Circuit}: Fixed-frequency\footnote{113kHz for 20mm disc} 555 timer circuit driving an amplifier/LC circuit generates an sinusoidal signal. Powers the piezoelectric disc.
    \end{itemize}
\end{itemize}

\clearpage

\subsubsection{Dehumidification}
\label{sec:dehumidification}

\textbf{Purpose}: Actively \textit{decreases} growth environment air humidity.

\textbf{Function}:
\begin{itemize}
    \item \textbf{Inputs}: Humid air (high water vapour content), dehumidification on/off control signal, dry desiccant
    \item \textbf{Outputs}: Dry air (low water vapour content), saturated desiccant, desiccant saturation level signal
\end{itemize}

\textbf{Method}:
\begin{enumerate}
    \item \textit{Setup}:
    \begin{enumerate}
        \item Connect dehumidification control signal to control module;
        \item Insert dry desiccant cartridge;
    \end{enumerate}
    \item \textit{Testing}:
    \begin{itemize}
        \item Desiccant removes moisture from air.
        \item Desiccant indicates saturation as expected, which is sensed by computer.
        \item Shutters operate as intended, and no dehumidification occurs when closed.
        \item Maximum dehumidification rate exceeds total plant transpiration rate.
    \end{itemize}
    \item \textit{Process}:
    \begin{enumerate}
        \item Dehumidification control signal activates fans and opens shutters;
        \item Humid air passes over the desiccant, and dry air exits the unit;
        \item Desiccant becomes saturated, and indicates degree of saturation;
        \item Indication is sensed by computer (\ref{sec:automation}), which notifies the user when to replace and dehydrate/"recharge" desiccant;
    \end{enumerate}
    \item \textit{Shutdown}:
    \begin{enumerate}
        \item Disconnect control signals from control module;
        \item Recharge cartridge;
    \end{enumerate}
\end{enumerate}

\textbf{Calculations}:\\
Assuming an air temperature of 30$\degree$C, water vapour saturation $p_{30C}$ of $30.4g/m^{3}$, relative humidity target range $[\text{\%RH}_{min},\text{\%RH}_{max}]$ of 20\% to 90\%, and 6\% dessicant capacity (by mass):
\begin{gather}
    m_{vapour} = \text{\%RH} * p_{30C} * V\\
    V_{4U} = (0.5m)^3 * 4 = 0.5m^{3} \\
    m_{extracted} = m_{max} - m_{min} = (\text{\%RH}_{max}-\text{\%RH}_{min}) * p_{30C} * V = 10.64\text{g water}\\
    m_{desiccant}=\frac{10.64g}{0.06 \%} = 177.3\text{g desiccant}
\end{gather}

$\therefore$ 177.3g of 6\% capacity desiccant is needed to change the RH\% of a 4U Class 2 setup from 90\% to 20\%.

\clearpage

\textbf{Features}:
\begin{itemize}
    \item \textit{Dehumidification Unit}: One input port and one output port. Comprised of:
    \begin{itemize}
        \item \textit{Fans}: Humidity-rated fans force moist air through the desiccant cartridge input port and dry air out of the output port.
        \item \textit{Filter}: Polyethylene-polyropylene blend (non-toxic) MERV 13 (0.3 micron) air filters \cite{filter} located at input and output ports of dehumidification chamber eliminate risk of any airborne pathogens being transferred onto silica beads and out of the system during cartridge recharging.
        \item \textit{Shutters}: Servo-actuated shutters enable opening and closing of dehumidifier input and output on demand. Air-tight when closed to prevent unintended dehumidification.
        \item \textit{Desiccant Cartridge}: Oven-safe. Easily removable for swapping and "recharging". Contains the silica gel desiccant.
        \item \textit{Indicating Silica Gel Desiccant}: Cheap, efficient, food-safe, reusable chemical desiccant beads with a water mass capacity of 6\% \cite{desiccant}. Changes color from blue to pink when saturated.
    \end{itemize}
    \item \textit{Color Sensor}: Optical color sensor \cite{colorsensor} senses cartridge saturation. Informs when to recharge the desiccant cartridge (see \ref{sec:automation}).
    \item \textit{Evaporator Oven}: A ventilated oven that can maintain 125°C for 12 hours \cite{desiccant}. Heats cartridge to evaporate/"bake off" moisture collected by silica beads, thus "recharging" them. Vapour is vented to onboard dehumidifier for recapture.
\end{itemize}

\clearpage

\subsection{Gas Composition and Exchange}
\label{sec:gas}

\textbf{Purpose}: Controls gas composition of the growth environment by mediating exchange with surroundings.

\textbf{Function}:
\begin{itemize}
    \item \textbf{Inputs}: Power, exchange control signal (fan rate and shutter open/close)
    \item \textbf{Outputs}: Gas intake (from surroundings), gas exhaust (to surroundings)
\end{itemize}

\textbf{Method}:
\begin{enumerate}
    \item \textit{Setup}:
    \begin{enumerate}
        \item Connect exhaust port to filtration/dehumidification system;
        \item Connect shutter servos, fans to control module;
    \end{enumerate}
    \item \textit{Testing}:
    \begin{itemize}
        \item Shutter servos, fans operate as intended;
        \item Shutters are sealed (air-tight) when closed;
        \item Exhaust filter removes all aerosols (i.e. pollen, seeds) and pathogens;
        \item Exhaust dehumidification brings humidity down to ambient (60\% on ISS); % TODO: Source?
    \end{itemize}
    \item \textit{Process}:
    \begin{enumerate}
        \item On-demand, both input and output ports activate. Shutters open, and fans are enabled;
        \item Input port draws in air from surroundings;
        \item Output port expels air through filtration and dehumidification system to be recycled;
    \end{enumerate}
    \item \textit{Shutdown}:
    \begin{enumerate}
        \item Disconnect exhaust port from filtration/dehumidification system;
        \item Disconnect shutter servos, fans from control module;
    \end{enumerate}
\end{enumerate}

\textbf{Features}:
\begin{itemize}
    \item \textit{Exchange Port}: Normally-sealed. Input and output. Each comprises:
    \begin{itemize}
        \item \textit{Exchange Fan}: Controls exchange rate. 
        \item \textit{Exchange Shutters}: Servo-controlled shutters allow for gas exhaust or intake for CO${}_2$/O${}_2$ regulation. Air-tight when closed.
    \end{itemize}
    \item \textit{Gas Concentration Sensors}: Collects data on concentrations (ppm) of various gasses (CO${}_2$, O${}_2$, etc.). Reports to automation (\ref{sec:automation}).
    \item \textit{Output Filter/Dehumidifier}: \textbf{Onboard life support systems} provides a dehumidifier and HEPA filter for removing microbes and reducing humidity.
\end{itemize}

\clearpage

\subsection{Lighting}

\begin{figure}[h!]
  \centering
  \includegraphics[width=\textwidth]{../assets/photos/prototype_lighting.png}
  \hfill
  \caption{Lighting prototype. Note 5 daisy-chained LED baords and driver board (rear-right), all mounted to lighting tray.}
  \label{fig:prototype_lighting}
\end{figure}

The prototype performs all functions as designed, except for UV lighting. A sixth LED series and LED driver module will be added in the next iteration.

\clearpage

\textbf{High Success Rates}: Complete automation and environmental control ensures high crop success rates and yield predictability.

\textbf{Repeatability}: Once optimal conditions are found for a given crop species, they can be repeated ad infinitum.

\textbf{Immediate Sensor Feedback and Response}: Immediate feedback from both environment sensors and plant metric analysis empowers the system to respond to unpredictable or otherwise uncontrolled factors (i.e. poor seed health, outside interference). Plant metric analysis can be used to diagnose program ineffectualities, accelerate optimization, and preventatively mitigate plant health decline.

\textbf{Data Collection, Yield Optimization}: By collecting data via computer vision and post-harvest yield evaluation (GCMS, weighing, etc.) on the plant's response to the induced environment, the relationship between the species behaviour and the surrounding environment can be analyzed. Plant metrics include plant health indicators (chlorophyll concentrations/chlorosis, leaf count/size distribution/density, plant height/canopy dimensions leaf tip burn, leaf curl, wilting, etc.) and crop yield (edible matter net mass/percent mass of plant, total plant mass, chemical/nutritional composition, caloric measurement, etc.). Data is filtered/smoothed across time to account for noise. The relationship is then represented by a statistical/machine learning model via a method known as "surrogate modelling". The method for this analysis can be found in the preliminary calculations Appendix \ref{app:optimization}.

\clearpage

% \section{System-Level Build Process Report}
% % System-level report (i.e. block diagram) of build process

% \subsection{Materials}
\subsubsection{System}

\textbf{Automation}\\

\begin{table}[!h]
    \centering
    \begin{tabular}{|c|l|l|l|c|}
    \hline
        Index   & Manufacturer Part Number  & Manufacturer Name         & Description                       & Quantity  \\ \hline
        1       & A000005                   & Arduino                   & ARDUINO NANO ATMEGA328 EVAL BRD   & 1         \\ \hline
        2       & S404GSEJ6-U3000-3         & "Delkin Devices, Inc."    & 4GB MLC MICROSD CARD (-25C - +85  & 1         \\ \hline
        3       & 61304021121               & Würth Elektronik          & CONN HEADER VERT 40POS 2.54MM     & 1         \\ \hline
        4       & SC0510                    & Raspberry Pi              & ZERO 2 W                          & 1         \\ \hline
        5       & DMN2005K-7                & Diodes Incorporated       & MOSFET N-CH 20V 300MA SOT23-3     & 2         \\ \hline
        6       & RC0603FR-0710KL           & YAGEO                     & RES 10K OHM 1\% 1/10W 0603        & 5         \\ \hline
        7       & 4484                      & Adafruit Industries LLC   & MINI PITFT 1.3 FOR RASPBERRY PI   & 1         \\ \hline
        8       & 5055670271                & Molex                     & CONN HEADER SMD R/A 2POS 1.25MM   & 2         \\ \hline
        9       & 5055670471                & Molex                     & CONN HEADER SMD R/A 4POS 1.25MM   & 5         \\ \hline
        10      & 5055670871                & Molex                     & CONN HEADER SMD R/A 8POS 1.25MM   & 3         \\ \hline
        11      & 5055670681                & Molex                     & CONN HEADER SMD R/A 6POS 1.25MM   & 3         \\ \hline
    \end{tabular}
    \caption{Automation system electronic components.}
    \label{tab:automation_components}
\end{table}

In addition, 1x \textit{Automation Motherboard PCB}: 2 Layers, 1 oz. Copper, 1.6mm Thickness, Suggested: HASL Finish (Lead-Free), White PCB, Black Silkscreen

\textbf{Housing}\\


\textbf{Aeroponics}\\


\textbf{Leaf-Zone Thermoregulation}\\


\textbf{Humidification}\\


\textbf{Dehumidification}\\


\textbf{Gas Composition Regulation and Exchange}\\


\textbf{Lighting}\\


\subsubsection{Inputs}
% Supply inputs (water, power, network), consumable inputs (pH/nutrient solutions, dehumidification cartridge)

\textbf{Supply Inputs}
\begin{itemize}
    \item \textit{Water}: reverse-osmosis, ambient
    \item \textit{Power}: 120V 60Hz AC\footnote{The power supply can be altered to suit a variety of power inputs (i.e. DC)}
    \item \textit{Network}: ethernet or wireless, optional
\end{itemize}

\textbf{Consumable Inputs}
\begin{itemize}
    \item \textit{Nutrient/pH Adjusment Solutions}: pouches
    \item \textit{Dehumidification Cartridge}: recharged
\end{itemize}

\subsubsection{Outputs}
% Food outputs(?), by-products/waste (waste water from flushing, dehumidification cartridge)

\textbf{Food Outputs}\\


\textbf{By-Products \& Waste}\\


\subsubsection{Maintenance}

\textbf{Spare Components}\\


\textbf{Tools}\\


\subsubsection{Cleaning}

\textbf{Soaps}\\


\textbf{Disinfectants}\\


\textbf{Tools}\\



\section{Prototype Build Status}
% SPECIFICALLY build status - from functional prototyping to testing to refinement

\subsection{Completion}
The prototype is 50\% complete as of May 31st, 2022. The housing, aeroponics supply system (without dosage pumps or thermoregulation), lighting, and automation systems are fully operational.


\subsection{Successes, Results, and Products}
% Please describe any successes (e.g. preliminary results) and predicted and/or produced food outputs

As of May 31st, 2022, no food has been produced by PeaPod. However, we predict successful production of a variety of plant-based food products over the coming weeks, derived from a variety of plant types (leafy greens, legumes, garden vegetables, microgreens, root vegetables, herbs) and prepared using a variety of methods and combinations (i.e. meals). This is possible despite prototype incompleteness, as the remainder of subsystems (except nutrient/pH adjustment solution dosage) are non-critical to plant growth when ambient temperature, humidity, and gas composition are relatively suitable for plant growth at typical indoor ambient atmospheric conditions. In addition, this will still produce usable data for optimization, as all sensors are in place and collecting data even though their associated control systems are absent.

\uline{\textbf{835 Characters} (5000 max)}

\subsection{Challenges}
% Please describe any challenges that you have encountered in developing your design and/or prototype since Phase 1.

Processes for pollination and germination are still unclear, though simple solutions are possible (air circulation for pollination, seed planting/germination in-place).

Due to the high current draw of the Peltier device (8.5A), as well as the necessity of voltage control over PWM, a custom driver circuit needed to be designed (as opposed to an off-the-shelf integrated circuit). The multi-stage driver (PWM to step-up MOSFETs to low-pass filter to operational amplifier voltage buffer to Darlington amplifier with feedback to relay H-bridge) was non-intuitive to design, and required many revisions (choosing Darlington over MOSFET due to op amp output current limits, choosing relays over MOSFETs due to variable supply voltage).

Specific component placement and orientation for extended housing topologies (as well as the process of actually extending a housing) is still unclear.

Selection of an aeroponics system type was a key design challenge, specifically around the ability to deliver mineral-rich water without calcification at high water delivery rates with suitable droplet diameter, which ultimately led to our selection of high-pressure nozzle-based aeroponics over piezoelectric mesh nebulizer aeroponics (as is used in some commercial settings).

Initially, the method of nutrient and pH-adjustment solution injection was Venturi siphon based. This presented a number of key issues, mainly controlling injection rate/mixing proportions while preventing backflow under pressure, which led us to pivot to peristaltic pumps, which have proven to be superior in both aspects.

\clearpage

The lighting system PCBs went through 4 revisions before consistently successful functionality was achieved. Reasons include pin alignment during LED board daisy-chaining, thermal management, and LED power limits.

The fitting selection for the aeroponics supply system was fully redesigned twice, first because of new component selection, and then again for part count/mass optimization. The aeroponics supply system also leaked frequently prior to optimizing the assembly process (proper securing of tapered and parallel thread mating faces using PTFE tape and bonded O-rings, respectively, as well as proper tightening).

The software has been restructured and portions completely rewritten countless times. Establishing and maintaining reliable computer-microcontroller communications over serial was challenging, as we did not initially realize that the two devices' IO operated at different voltages (3.3V vs 5V, respectively). Ensuring message validity (JSON formatting, delimiters, encoding, etc.) was also difficult, and was ultimately solved by writing unit testing suites for both devices' communication software. In addition, being able to flash microcontroller firmware from the computer on the fly required a number of revisions. 

Cloud communication was also a major challenge, as registering an IoT device with an API required keypair generation. Manual/hard-coded keypairs are not user-friendly or particularly secure, and a cryptographic integrated circuit built into the motherboard goes against our open-source mission. We ultimately settled on the following process: the user securely enters their credentials for an authentication provider (Google, GitHub, Microsoft, etc.) using the OAuth Device Flow method, and the device "logs in" to our API as the user. The device can then request a private key from our API, which generates the keypair (supposing the user has not exceeded their device quota) and stores the public key on the backend. The device has effectively "self-registered", and can publish data and recieve live config data autonomously.

For a long time, the aeroponic container was a "black box" in our design, meaning that we had no idea where to even begin. One of our team members suggested tent material, and prototyping took off once we found the specific composite that met our requirements (high strength/low weight, fully waterproof, no coating/food-safe, relatively easy to assemble). Assembling the container required a number of iterations, as we initially overlooked the additional material needed for seams (fused with pressure-sensitive adhesive) when calculating the dimensions, as well as choosing a variant of the composite fabric that was too thin and ultimately developed microtears.

Logistical issues also presented themselves during the prototyping process. We had to move locations twice, and our primary 3D printer spontaneously broke.

Currently, the custom peristaltic dosage pumps are being prototyped. This requires tight tolerances in 3D printing, due to the specific mechanism of peristalsis (pump housing not tight enough, tubing does not fully shut under force from the rollers and water is able to backflow through the pump; pump housing is too tight, and the tubing wears out very quickly, or the rollers don't move at all, overheating the stepper motor). In addition, the thermoregulation systems (both leaf- and root-zone) are nearing completion, with only heat sink selection and subsystem testing remaining.

\uline{\textbf{5108 Characters} (5000 max)}

\clearpage

\subsection{Timeline}
% What are your next steps? Incl. estimated completion date

% maximum 5000 characters

The timeline as of May 31st 2022 is as follows:
\begin{itemize}
    \item \textit{June 2022}: Finish design. Continue prototyping. Grow our first plants and collect data.
    \item \textit{July 2022}: Complete prototype. Assemble and distribute prototypes to schools and volunteers for beta testing, publicly sharing collected data and diagnostic information. Make design/prototype improvements based on beta testing. Begin constructing the optimization machine learning model.
    \item \textit{August 2022}: Collect samples for nutritional and safety analysis. Collaborate globally on distributed phenological research. Publish findings.
\end{itemize}

Estimated latest prototype completion date: \textbf{August 1st, 2022}.

\clearpage

\section{Prototype Progress}
% To be extracted as a separate PDF and submitted as 2.5

\subsection{Automation}
\label{sec:automation}

\textbf{Purpose}: Performing growth-, maintenance-, and data-related tasks autonomously on the basis of both schedule and necessity to reduce crew maintenance time, improve consistency of products, and eliminate safety risks. Maintains the homogeneity of the internal environment with increased accuracy and precision over crew interference, while enabling simultaneous control over all parameters.

\textbf{Function}:
\begin{itemize}
    \item \textbf{Inputs}: Environment data stream (sensor readings), growth program
    \item \textbf{Outputs}: Actuator control signals, crew/cloud messaging, environment data (stored), timelapse photo set
\end{itemize}

\textbf{Method}:
\begin{enumerate}
    \item \textit{Setup}:
    \begin{enumerate}
        \item Power is connected and system is booted;
        \item Program is selected by user;
    \end{enumerate}
    \item \textit{Testing}:
    \begin{itemize}
        \item Power-on Self-Test (POST) passes (i.e. all hardware is online and communicating as expected);
        \item Systems enact program as intended (i.e. control systems respond properly);
    \end{itemize}
    \item \textit{Process}:
    \begin{enumerate}
        \item Checks operating preconditions (self POST and per-subsystem);
        \item \textbf{Environment Control Loop} (matches \textit{Sense-Plan-Act} model of robotics): %TODO cite the model
        \begin{enumerate}
            \item \textit{Sense}: Receives and stores data about current environment state;
            \item \textit{Plan}: Compares current state to "desired"/program state, develops a "plan"/actuator control to reach desired state;
            \item \textit{Act}: Controls subsystem operations in order to enact the plan;
        \end{enumerate}
        \item Notifies user on maintenance requirement (i.e. non-automated input/output management, refills, repairs, etc.), end-of-program (EOP), and diagnostic info;
    \end{enumerate}
    \item \textit{Shutdown} (either manual or EOP):
    \begin{enumerate}
        \item Stop all subsystem operations;
        \item Power down;
    \end{enumerate}
\end{enumerate}

\textbf{Features}:

A dual-computer system was chosen, as this allowed for discrete management of high-level functionality (camera capture, internet/cloud functionality, local storage, complex calculations, etc.) and low-level functionality (hardware-level communications, actuator control and GPIO) in a master/slave topology with a constant two-way stream of shared data.

\begin{itemize}
    \item \textit{Computer (Master)}: Manages data collection, batch storage, analysis, and transmission/receiving, as well as planning/calculations for actuator control. Includes internal clock (for program, notification), network connection (for data transmission, notification), photo capture, and non-volatile storage (for data/photos). Sends instructions derived from the program to the microcontroller.
    \item \textit{Microcontroller (Slave)}: Manages detailed sensor/actuator states and communications with them (on/off, sensor readings, actuator control). Streams collected sensor readings to the computer.
    \item \textit{Program}: Set of actuated instructions (e.g. lights on) and control targets (e.g. hold air temperature at 22°C) to enact at specific points in the growth cycle, as well as config data;
    \item \textit{Camera Capture \& Plant Performance Metric (PPM) Extraction}: Top-down and side-view cameras, captured under standard lighting at regular intervals throughout the course of the growth cycle. For live feed transmission to users (local and remote), as well as PPM extraction via computer vision and machine learning for yield and diagnostic analysis. Potentially relevant PPMs include (but are not limited to):
    \begin{itemize}
        \item Leaf health indicators (i.e. leaf tip burn, leaf curl, chlorosis);
        \item Leaf count, size distribution;
        \item Leaf density;
        \item Canopy dimensions/surface area;
        \item Plant height;
        \item Fruit/harvest body size, ripeness;
    \end{itemize}
    \item \textit{Environment Data}: Record the environment's current state. Covers each \textit{control loop} environment parameter (e.g. included in a feedback loop). Control loop environment parameters include:
    \begin{itemize}
        \item Leaf-zone temperature (see \ref{sec:airthermoregulation});
        \item Leaf-zone humidity (see \ref{sec:humidityregulation});
        \item Root-zone temperature (see \ref{sec:aeroponics});
        \item Gas concentrations (see \ref{sec:gas});
    \end{itemize}
    \item \textit{Actuator Control}: Induces change in environment parameters. Covers both \textit{control loop} and \textit{actuated instruction} environment parameters. Actuated instruction environment parameters include:
    \begin{itemize}
        \item Lighting (see \ref{sec:lighting});
        \item Water delivery (see \ref{sec:aeroponics});
        \item Plant nutrient delivery (see \ref{sec:aeroponics});
        \item Water pH (see \ref{sec:aeroponics});
        \item Air circulation rate (see \ref{sec:airthermoregulation});
    \end{itemize}
    \item \textit{Diagnostic Systems}: Include informative sensors tracking system input availability, subsystem diagnostics, etc. as well as notification triggers.
\end{itemize}

\clearpage

\subsection{Housing}
\label{sec:housing}

\textbf{Purpose}: \textit{Isolates} and \textit{insulates} growth environment from surroundings (heat, light, water vapour, air). Provides structural integrity and mounting points for other subsystems. Enables system extendability via repeated "unit cell" topology.

\textbf{Method}:
\begin{enumerate}
    \item \textit{Setup}:
    \begin{enumerate}
        \item Construct frame and install panels;
        \item Mount control module (w/ subsystems), connect inputs and internal subsystem connections;
        \item Install tray mounts, insert trays (w/ subsystems);
    \end{enumerate}
    \item \textit{Testing}:
    \begin{itemize}
        \item Frame construction is rigid, level, and sturdy;
        \item Panels are insulating against temperature changes;
    \end{itemize}
    \item \textit{Process}:
    \begin{enumerate}
        \item Panels insulate against heat gain/loss, are opaque, and contain light and heat via reflection;
        \item Shell construction is tight, thus sealing against moisture;
        \item Internal vertical mounting channels for systems, horizontal plane "trays";
        \item \textbf{Extension} (can be repeated):
        \begin{enumerate}
            \item Add a second housing;
            \item Remove dividing panel from both housings;
            \item Remove "shared" skeleton extrusions from second housing;
            \item Join the two housings to form one larger 2x1 housing;
            \item \textbf{Extension Modes} (may be combined in any way to suit application):
            \begin{itemize}
                \item \textit{Option 1} (Smaller Housings): Operate the combined housing off \textbf{one} control module.
                \item \textit{Option 2} (Larger Housings): Add control modules to account for additional air volume, plant count, power requirement, etc.. Operate in a \textbf{controller-follower topology}.
                \item \textit{Option 3} (Frame Connection Only): Leave the dividing panel, add a control module, and operate the two PeaPods \textbf{separately}.
            \end{itemize}
        \end{enumerate}
    \end{enumerate}
    \item \textit{Shutdown}:
    \begin{enumerate}
        \item Dismount all systems, remove trays;
        \item Disassemble housing;
    \end{enumerate}
\end{enumerate}

\textbf{Features}:
\begin{itemize}
    \item \textit{Frame}: T-slotted aluminum extrusion framing with aluminum face-mounted brackets forms a cubic skeleton for rigidity/strength (high strength-to-weight aluminum) and easy component mounting and repositioning (standard mounting channels). These extrusions form the "edges" of the cubic housing. %Todo: cite extrusion
    \item \textit{Panels}: Graphite-enhanced expanded polystyrene (GPS) rigid foam insulation panels \cite{insulation} with reflective mylar internal lamination increase energy efficiency (GPS RSI of 0.0328$\frac{m^2 \cdot \degree C}{W}$ per mm of thickness, mylar enables light/heat reflection), as well as safety against cross-contamination and pathogens. Panels slide into extrusion channels and form a "seal" for greater water vapour retention. These panels form the "faces" of cube. %Todo: cite mylar
    \item \textit{Trays}: Horizontal plane subframes mounted to internal vertical extrusion channels for ease of repositioning. Trays slide in/out on permanent mounts. All connections are quick-connect (i.e. quick-diconnect tubing for grow tray, push connectors for lighting) for ease of removal. Trays include:
    \begin{itemize}
        \item \textit{Grow Trays}: Support plants (via grow cups), aeroponic nozzles, aeroponics container, and supply/recycling lines (See \ref{sec:aeroponics}).
        \item \textit{Lighting Trays}: Support LED boards, driver board (See \ref{sec:lighting}).
    \end{itemize}
\end{itemize}

\clearpage

\subsection{Aeroponics}
\label{sec:aeroponics}

\textbf{Purpose}: Delivers plant nutrients and pH- and temperature-controlled water to the roots via a fine mist.

\textbf{Function}:
\begin{itemize}
    \item \textbf{Inputs}: Reverse osmosis water\footnote{RO water has no dissolved nutrients and a neutral pH of 7.0. This enables easier and more reliable calculations. In addition, it has no particulate or minerals, minimizing the chances of nozzle clog.} under positive pressure, concentrated pH up \& down solutions and nutrient solutions, nozzle delivery on/off control (\ref{sec:automation}), pH and nutrient solution ratios as control signals (dosing pump speeds; \ref{sec:automation}), water thermoregulation control signal (\ref{sec:automation})
    \item \textbf{Outputs}: pH- and nutrient-controlled water mist (50 micron mean droplet diameter)
\end{itemize}

\textbf{Method}:
\begin{enumerate}
    \item \textit{Setup}:
    \begin{enumerate}
        \item Hook up water, solution, and signal inputs;
        \item Connect the quick-disconnect fitting;
        \item Calibrate pressure, temperature sensors to atmospheric;
        \item Enable water input to prime system (if known pressure/temperature, calibrate sensors);
        \item Mount container, connect runoff collection line to recycling port;
    \end{enumerate}
\newpage
    \item \textit{Testing}:
    \begin{itemize}
        \item Temperature, pressure sensors communicate as expected;
        \item No leaks at any connections under a) source pressure, b) fully pressurized;
        \item Pump actuates and auto-shuts off as expected, and is able to deliver the required pressure;
        \item All components, tubing, and connectors/fittings withstand full pressurization;
        \item Solenoid is normally closed, withstands full pressurization, and opens when power is applied;
        \item Quick-disconnect operates as intended at full pressurization without leaks;
        \item Nozzles produce even-distribution full-cone mist;
        \item Manual and actuated valves operate as intended;
        \item Runoff container is sealed, and runoff collection operates as intended;
    \end{itemize}
    \item \textit{Process}:
    \begin{enumerate}
        \item Water is pressurized to constant 80psi;
        \item Heat is added to or removed from the water;
        \item Temperature and pressure of the water is read (feedback);
        \item Nutrient and pH solutions are mixed in-line at an adjustable ratio\footnote{I.e. add X mL of nutrient solution Y per mL water to achieve Z ppm, or add A mL of pH down solution per mL water to achieve a pH of B.};
        \item Flow to nozzle is controlled (on/off);
        \item Nozzle turns pressurized water into mist;
        \item Runoff is contained by a water-tight container, and collected for recycling;
    \end{enumerate}
    \item \textit{Shutdown}:
    \begin{enumerate}
        \item Power down the pump and thermoregulation unit;
        \item Close the nutrient and pH solution valves;
        \item Close the source shutoff valve;
        \item Open the drain valve, and allow the system to depressurize completely;
        \item Re-open the source shutoff valve and flush the system with fresh water;
        \item Power down the solenoid;
        \item Collect all remaining runoff;
        \item Disconnect the quick-disconnect fitting;
        \item Disconnect the inputs;
    \end{enumerate}
\end{enumerate}

\textbf{Features} (in order of plumbing; source $\to$ nozzle):
\begin{itemize}
    \item \textit{Water Source}: Input for ambient reverse-osmosis water.
    \item \textit{Manual Source Shutoff Valve}: Ball valve.
    \item \textit{Diaphragm Pump}: Self-priming, auto-shutoff at 80psi. Power is controlled by a relay.
    \item \textit{Inline Thermoelectric Water Heater/Cooler Block}: Aluminum water block heat pump. See Section \ref{sec:airthermoregulation}.
    \item \textit{Solution Injection Manifold}: A manifold of parallel inline injectors, allowing for on-demand adjustment of mixing ratios for nutrient and pH solutions. Comprises:
    \begin{itemize}
        \item \textit{Manifold}: Splits the water line into a set of parallel branches with inline tees to enable solution injection.
        \item \textit{Dosing Pumps}: Stepper-motor driven custom peristaltic pumps deliver solutions at a controlled rate/ratio (one per solution). Toleranced to prevent backflow at pressure. % TODO more details (tubing type, washers, bearings, etc.) w/ part numbers
        \item \textit{Nutrient Solutions}: Aqueous. Highly concentrated. Selectable as part of the program (\ref{sec:automation})\footnote{Many different solutions can be combined (according to solubility laws, pH requirements, etc.).}, and may include any of:
        \begin{itemize}
            \item Bioavailable nonmetals (ammonia, ammonium, nitrates, nitrites, phosphates, sulfates, etc.)
            \item Bioavailable metals (potassium, etc.)
            \item Minerals (magnesium, calcium)
            \item Other trace elements
            \item Custom solutions (i.e. fungicides/algicides, descaling solutions)
        \end{itemize} 
        \item \textit{pH Adjustment Solutions}\footnote{\textit{NOTE:} Ionic composition of pH solutions should be considered in the understanding of the nutrient composition (i.e. phosphic acid results in phosphate ions in spray)}: Aqueous. Highly concentrated. One for pH up (>8), one for pH down (<6).
        \item \textit{Solution Storage Containers}: Opaque, insulated, chemical-safe, refillable cartridges. Prevent degradation of solution compounds over time via light or heat.
        \begin{itemize}
            \item \textit{Fill Level Sensors}: Depth sensors measure fill level of container. Notifies user to refill.
        \end{itemize}
    \end{itemize}
    \item \textit{Water Temperature Sensor}: Tee-fitted. Informs a \textbf{PID control loop}. See Section \ref{sec:airthermoregulation}.
    \item \textit{Accumulator Tank}: Uses an air bladder to maintain and stabilize pressure.
    \item \textit{Pressure Sensor}: Allows for shutoff of pump in case of emergency.
    \item \textit{Drain Valve}: Tee-fitted ball valve. Allows the system to be depressurized and drained.
    \item \textit{Solenoid Valve}: Controls delivery to the nozzles to enable on-demand misting.
    \item \textit{Grow Tray Quick-Disconnect}: Connectors between aeroponics supply and nozzles that allow for quick disconnection with auto-shutoff so the trays may be removed.
    \item \textit{Nozzle}: Mounted to grow tray, pointed at plant roots. 80psi water through a 0.4-0.6mm orifice produces 5-50 micron water droplets, optimal for plant growth. This method is 98\% more water-efficient than traditional farming.%TODO: Sources??
    \item \textit{Root-Zone Container}: Watertight container that encapsulates the entire root zone. Made of a woven waterproof composite fabric (CT5K.18 mylar with Dyneema, 1.43oz/yd${}^2$), chosen for high strength-to-weight ratio (15x that of steel) and natural no-coating food-safe waterproof quality \cite{dyneema}. Mounted and \textbf{sealed} to the grow tray with a drawstring for easy root zone access. Provides water supply and runoff collection ports.
\end{itemize}

\clearpage

\subsection{Lighting}

\begin{figure}[h!]
  \centering
  \includegraphics[width=\textwidth]{../assets/photos/prototype_lighting.png}
  \hfill
  \caption{Lighting prototype. Note 5 daisy-chained LED baords and driver board (rear-right), all mounted to lighting tray.}
  \label{fig:prototype_lighting}
\end{figure}

The prototype performs all functions as designed, except for UV lighting. A sixth LED series and LED driver module will be added in the next iteration.

\clearpage

% References
\bibliographystyle{IEEEtran}
\bibliography{references}
\end{document}